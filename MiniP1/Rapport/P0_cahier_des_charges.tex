\chapter{Cahier des charges}
\paragraph{Introduction :}
 L'objectif est de définir un outil de simulation  d'ordonancement de tâche en temps réel. Parmies ses fonctionalités, l'outil devra pour tester les contraintes temporelles d'un ensemble de tâches générées au préalable. La génération de ces tâches entre par dans la conception de l'outil. Cet outil permettra d'exporter le résultat dans un fichier  d'extension $.ktr$ pour être exploité directement par l'outil graphique Kiwi.
 
\section{Données en entrées}
L'outil doit pouvoir permettre à l'utilisateur de rentrer des tâches périodiques et ou apériodiques lui même (en précisant chacune des attributs) ou de demander une génération aléatoire pour les deux catégories.
\section{Fonctionnement}
\begin{itemize}
\item
Une analyse d'ordonnançabilité. L'outil affichera à l'utilisateur les résultats des différents test  s'afficheront avec les conclusions qui en découlent.
\item
Un environnement de simulation. L'outil lors du calcul de de l'ordonancement devra afficher les différents événements. Un bilan de ces action sera résumé dans un fichier au terme de l'execution (facultatif).

L'outil doit pouvoir proposer plusieurs politiques d'ordonnancement. \'A savoir : 
\begin{itemize}
\item
Pour les tâches périodiques :

\begin{itemize}
\item
Rate Monotonic
\item
EDF
\end{itemize}

\item
Pour les tâches apériodique : 
\begin{itemize}
\item
BG
\item
TBS
\end{itemize}

\end{itemize} 
\item
Un fichier d'extension $.ktr$ sera généré au terme de l'execution, et contiendra le déroulement de l'ordonancement jusqu'a son terme.
\item
L'outil doit communiquer au terme de l'execution différents résusltats de performance qu'il aura lui même calculés. Les informations à fournir sont les suivantes : 
\begin{itemize}
\item
Le nombre de violations d'échéances.
\item
le nombre de commutations de contexte et de préemptions.
\item

\end{itemize}
\end{itemize}

