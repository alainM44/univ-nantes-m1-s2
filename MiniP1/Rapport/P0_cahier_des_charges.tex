\chapter{Introduction}
\paragraph{Introduction :}
 L'objectif est de définir un outil de simulation  d'ordonancement de tâche en temps réel. Parmies ses fonctionalités, l'outil devra pour tester les contraintes temporelles d'un ensemble de tâches générées au préalable. La génération de ces tâches entre par dans la conception de l'outil. Cet outil permettra d'exporter le résultat dans un fichier  d'extension $.ktr$ pour être exploité directement par l'outil graphique Kiwi.
 
\section{Format d'entrée}
Il est définit dans le langage XML. Sa syntase sera la suivante
\subsection{Entête}
\begin{itemize}
\item

Des balises \verb+<genTache.AbstractTache-array>+ encadrent la totalité du fichier.
\item
Une tâche périodique sera définie dans une balise  \verb+<genTache.TachePeriodique>+ 
\item
Une tâche apériodique sera définie dans une balise  \verb+<genTache.TacheAPeriodique>+ 
\item
Dans une tache tous ses attributs seront définis de la manière suivante \verb+<nom_attribut>valeur_attribut</nom_attribut>+
\end{itemize}

\section{Fonctionnement}
\begin{itemize}
\item
Une génération des tâches dans le format définit ci-dessus. Cette génération doit pouvoir être aléatoire ou définie entièrement par l'utlisateur.
\item
Une analyse d'ordonnançabilité. L'outil affichera à l'utilisateur les résultats des différents test  s'afficheront avec les conclusions qui en découlent.
\item
Un environnement de simulation. L'outil lors du calcul de de l'ordonancement devra afficher les différents événements. Un bilan de ces action sera résumé dans un fichier au terme de l'execution (facultatif).
\item
Un fichier d'extension $.ktr$ sera généré au terme de l'execution, et contiendra le déroulement de l'ordonancement jusqu'a son terme.
\end{itemize}

