\chapter{Point de vue Utilisateur}
\chaptermark{Utilisateur}

\section{Plugins développés}
\sectionmark{Plugins}
Middleware qui est notre plugin principal propose la vue Eclipse suivante : 
  \begin{figure}[h]
	  \centering
	  \includegraphics[scale=0.50]{img/Capture}
	  \caption{Vue du plugin}
	  \label{fig:vue}
\end{figure} 
\begin{description}
\item[Load Video]
L'utilisateur doit au préalable avoir renseigné le chemin de la vidéo à analyser. Le plugin Acquisition est sollicité au déclenchement de ce bouton. 
\item[Lancer Analyse]
Appel du plugin Reasoning qui effectue une analyse de la vidéo précedement acquise.
\item[Post]
L'utilisateur doit au préalable avoir renseigné l'acces token nécessaire pour poster sur une plateforme en ligne. 
Le plugin en charge de poster sur un blog ou un réseau social est appelé.
\end{description}
 
 \clearpage
 
\section{Lancement et manuel}
Pour lancer le plugin suivez les étape suivante:

\begin{enumerate}
 \item Cliquez dans le menu sur \og Run \fg, puis et cliquez sur \og Run configuration \fg, 
afin d'offrir un fenêtre de configuration (cf : \ref{fig:fenetreConfg} ).
  \item Une fois la fenêtre ouverte cliquez sur l'onglet plugin et cochez les label:
  \begin{enumerate}
      \item Acquisition
      \item Middleware
      \item NetP
      \item Reasoning
      \end{enumerate}
      \item Une fois les paramètres configurés cliquer sur \verb+Run+.
      \item Attendez que \emph{Eclipse} se lance. Lorsqu'\emph{Eclipse} sera lancé, allez dans \verb+Window+ $\rightarrow$ \verb+Show View+ $\rightarrow$ \verb+Other+\dots
      \item Choisissez \verb+middleware_view+ puis la vue \verb+Middleware+ (cf : \ref{fig:ShowView})
\end{enumerate}
  \begin{figure}[h]
	  \centering
	  \includegraphics[scale=0.40]{img/tuto}
	  \caption{Fenêtre de configuration}
	  \label{fig:fenetreConfg}
\end{figure}
  \begin{figure}
	  \centering
	  \includegraphics[scale=0.50]{img/ShowView}
	  \caption{Fenêtre Show View}
	  \label{fig:ShowView}
\end{figure}



