\chapter{Génération de boîtes}
Pour la génération des boîtes le nombre de caractéristiques est fixé à 20 et la dimension du pavage à 100.

\section{Map}

\subsection{Une première proposition de structure}
Pour les Maps l'organisation des structures de données est la suivante :
\begin{itemize}
\item Il n'y a aucune structure de données globale, tout est stocké au niveau de la boîte
 \item Chaque boîte contient quatre Maps contenant les différentes informations :
\begin{itemize}
 \item Une Map pour l'ensemble des intervalles coordonnées.
\item Une Map pour les caractéristiques de type String.
\item Une Map pour les caractéristiques de type Interval.
\item Une Map pour les caractéristiques de type Number.
\end{itemize}
\end{itemize}

\subsubsection{Map sans valeur d'initialisation}
Les résultats obtenus sont indiqués dans la table \ref{tab:hashmap1} et \ref{tab:treemap1}


\subsubsection{Map avec valeur d'initialisation}
Les résultats obtenus sont indiqués dans la table \ref{tab:hashmap2}


Un TreeMap ne peut avoir de paramètre de taille fourni à l'initialisation.

\paragraph{} On peut observer que l'initialisation des Maps n'influence pas sur leur temps de construction ou leur taille.

\subsection{Une seconde proposition de structure}
Pour les map, une autre organisation des structures de données possible est la suivante :
\begin{itemize}
\item Il y a un TreeMap globale contenant le type de chaque caractéristique.
 \item Chaque boîte contient deux Maps contenant les différentes informations :
\begin{itemize}
 \item Une map pour l'ensemble des intervalles coordonnées.
\item Une liste pour l'ensemble des caractéristiques stockées dans la boîte sous la forme d'Objects
\end{itemize}
\end{itemize}

Les résultats obtenus sont indiqués dans la table \ref{tab:hashmap3} et \ref{tab:treemap2}



\section{Liste et Tableau}
Pour les structures liste et tableau l'organisation des structures de données est la suivante :
\begin{itemize}
\item Il y a un TreeMap globale contenant le type de chaque caractéristique.
 \item Chaque boîte contient deux listes (ou tableaux) :
\begin{itemize}
 \item Une liste pour l'ensemble des intervalles coordonnées.
\item Une liste pour l'ensemble des caractéristiques de la boîte stockées sous la forme d'Objects
\end{itemize}
\end{itemize}

Les résultats obtenus sont indiqués dans la table \ref{tab:arraylist} et \ref{tab:linkedlist}


\section{Maps globales}
Les boîtes sont composées  de 4 ArrayLists : 
\begin{itemize}
  \item Une ArrayList pour l'ensemble des intervalles coordonnées.
  \item Une ArrayList pour les caractéristiques de type String.
  \item Une ArrayList pour les caractéristiques de type Interval.
  \item Une ArrayList pour les caractéristiques de type Number.
\end{itemize}
Il faut cependant savoir à quels indices se trouves les différents élément pour effectuer des accès directs. Pour cela, on dispose ici de quatre Maps globales :
\begin{itemize}
  \item Une Map pour l'ensemble des intervalles coordonnées.
  \item Une Map pour les caractéristiques de type String
  \item Une Map pour les caractéristiques de type Interval
  \item Une Map pour les caractéristiques de type Number
\end{itemize}
La composition de chacune de ces maps est la suivante :  
\begin{description}
 \item[Clef :]
ID de la coordonée ou de la variable.
\item[Valeur :]
Indice de l'objet dans le tableau de la boîte.
\end{description}


\paragraph{Observations :}
On propose trois cas de tests ou les Maps globales seront :
\begin{itemize}
  \item Des HashMaps :  \ref{tab:hashmapglobal} et \ref{tab:hashmapglobalInit}
  \item Des TreeMaps :  \ref{tab:treemapglobal} et \ref{tab:treemapglobalInit}
  \item Une TreeMaps construites à partir des HashMaps correspondantes : \ref{tab:treehashmapglobal} et \ref{tab:treehashmapglobalInit}
\end{itemize}
 Pour chaque cas, nous avons recommencé la série de tests en donnant aux constructeurs le nombre d'éléments qu'ils vont contenir. Lorsque cette information est fournie au constructeurs, les résultats révèlent dans les trois cas que la quantité de mémoire allouée est plus importante, mais celle réellement utilisée est plus faible. Par ailleurs on ne constate aucune différence flagrantes entre les trois structures. Pour 100000 boîtes elles nécessitent toutes les trois entre 310Mo et 350Mo. Les différences se manifesteront en théorie  en cas d'accès à des boîtes.
 
\appendix
\chapter{Annexes}
\section{Études de performance avec des Maps}

\begin{table}[htpb]
  \centering
\begin{tabular}{|c|c|c|c|c|}
\hline
Nombre de boîtes & Temps écoulé & CPU & Mémoire totale & Mémoire utilisée\\
\hline
100 & 0.02687s & 0.05s & 52Mo & 1Mo\\
\hline
1000 & 0.05415s & 0.09s & 52Mo & 7Mo\\
\hline
10000 & 0.36794s & 0.16s & 119Mo & 70Mo\\
\hline
100000 & 10.50770s & 0.86s & 776Mo & 667Mo\\
\hline
1000000 & \multicolumn{4}{|c|}{Out of memory}\\
\hline
\end{tabular}
\caption{Étude de performances avec des HashMap sans initialisation}
\label{tab:hashmap1}
\end{table}

\begin{table}[htbp]
  \centering
\begin{tabular}{|c|c|c|c|c|}
\hline
Nombre de boîtes & Temps écoulé & CPU & Mémoire totale & Mémoire utilisée\\
\hline
100 & 0.02873s & 0.06s & 52Mo & 1Mo\\
\hline
1000 & 0.05472s & 0.07s & 52Mo & 6Mo\\
\hline
10000 & 0.48741s & 0.16s & 138Mo & 64Mo\\
\hline
100000 & 7.47501s & 0.84s & 776Mo & 635Mo\\
\hline
1000000 & \multicolumn{4}{|c|}{Out of memory}\\
\hline
\end{tabular}
\caption{Étude de performances avec des TreeMap sans initialisation}
\label{tab:treemap1}
\end{table}

\begin{table}[htbp]
  \centering
\begin{tabular}{|c|c|c|c|c|}
\hline
Nombre de boîtes & Temps écoulé & CPU & Mémoire totale & Mémoire utilisée\\
\hline
100 & 0.02125s & 0.06s & 52Mo & 1Mo\\
\hline
1000 & 0.04448s & 0.07s & 52Mo & 7Mo\\
\hline
10000 & 0.38182s & 0.15s & 123Mo & 69Mo\\
\hline
100000 & 10.61614s & 0.81s & 776Mo & 667Mo\\
\hline
1000000 & \multicolumn{4}{|c|}{Out of memory}\\
\hline
\end{tabular}
\caption{Étude de performances avec des HashMap avec initialisation}
\label{tab:hashmap2}
\end{table}

\begin{table}[htbp]
  \centering
\begin{tabular}{|c|c|c|c|c|}
\hline
Nombre de boîtes & Temps écoulé & CPU & Mémoire totale & Mémoire utilisée\\
\hline
100 & 0.02201s & 0.06s & 52Mo & 1Mo\\
\hline
1000 & 0.04329s & 0.08s & 52Mo & 7Mo\\
\hline
10000 & 0.35066s & 0.13s & 118Mo & 69Mo\\
\hline
100000 & 9.99475s & 0.84s & 776Mo & 655Mo\\
\hline
1000000 & \multicolumn{4}{|c|}{Out of memory}\\
\hline
\end{tabular}
\caption{Étude de performances avec des HashMap seconde structure}
\label{tab:hashmap3}
\end{table}



\begin{table}[htbp]
  \centering
\begin{tabular}{|c|c|c|c|c|}
\hline
Nombre de boîtes & Temps écoulé & CPU & Mémoire totale & Mémoire utilisée\\
\hline
100 & 0.02792s & 0.07s & 52Mo & 1Mo\\
\hline
1000 & 0.05299s & 0.08s & 52Mo & 6Mo\\
\hline
10000 & 0.50868s & 0.15s & 138Mo & 64Mo\\
\hline
100000 & 7.54041s & 0.88s & 776Mo & 633Mo\\
\hline
1000000 & \multicolumn{4}{|c|}{Out of memory}\\
\hline
\end{tabular}
\caption{Étude de performances avec des TreeMap seconde structure}
\label{tab:treemap2}
\end{table}
\clearpage
\section{Études de performance avec des Tableaux et des Listes}

\begin{table}[h]
  \centering
\begin{tabular}{|c|c|c|c|c|}
\hline
Nombre de boîtes & Temps écoulé & CPU & Mémoire totale & Mémoire utilisée\\
\hline
100 & 0.018991s & 0.05s & 52Mo & <1Mo\\
\hline
1000 & 0.03237s & 0.06s & 52Mo & 4Mo\\
\hline
10000 & 0.11064s & 0.09s & 66Mo & 37Mo\\
\hline
100000 & 1.85945s & 0.30s & 408Mo & 342Mo\\
\hline
1000000 & \multicolumn{4}{|c|}{Out of memory}\\
\hline
\end{tabular}
\caption{Étude de performances avec des ArrayList}
 \label{tab:arraylist}
\end{table}

\begin{table}[htbp]
  \centering
\begin{tabular}{|c|c|c|c|c|}
\hline
Nombre de boîtes & Temps écoulé & CPU & Mémoire totale & Mémoire utilisée\\
\hline
100 & 0.02318s & 0.05s & 52Mo & 1Mo\\
\hline
1000 & 0.05196s & 0.08s & 52Mo & 5Mo\\
\hline
10000 & 0.21947s & 0.1s & 88Mo & 54Mo\\
\hline
100000 & 6.12441s & 0.30s & 776Mo & 542Mo\\
\hline
1000000 & \multicolumn{4}{|c|}{Out of memory}\\
\hline
\end{tabular}
\caption{Étude de performances avec des LinkedList}
\label{tab:linkedlist}
\end{table}
\clearpage




\section{Études de performance avec des maps globales}
%hashmap
\begin{table}[h]
  \centering
\begin{tabular}{|c|c|c|c|c|}
\hline
Nombre de boîtes & Temps écoulé & CPU & Mémoire totale & Mémoire utilisée\\
\hline
100 & 0.00646& 0.05s & 52Mo & 0Mo\\
\hline
1000 & 0.04284 & 0.07s & 52Mo & 4Mo\\
\hline
10000 & 0.146819 & 0.1s & 66Mo & 38Mo\\
\hline
100000 &2.071468 & 0.34s & 424Mo & 351Mo\\
\hline
1000000 & \multicolumn{4}{|c|}{Out of memory}\\
\hline
\end{tabular}
\caption{Étude de performances avec des HashMap globales sans initialisation} 
\label{tab:hashmapglobal}
\end{table}


\begin{table}[h]
  \centering
\begin{tabular}{|c|c|c|c|c|}
\hline
Nombre de boîtes & Temps écoulé & CPU & Mémoire totale & Mémoire utilisée\\
\hline
100 & 0.00237& 0.05s & 52Mo & 0Mo\\
\hline
1000 & 0.03380 & 0.07s & 52Mo & 3Mo\\
\hline
10000 & 0.19556 & 0.09s & 89mo & 32mo\\
\hline
100000 & 2.0883 & 0.25s & 480mo & 312mo\\
\hline
1000000 & \multicolumn{4}{|c|}{Out of memory}\\
\hline
\end{tabular}
\caption{Étude de performances avec des HashMap globales avec initialisation}
\label{tab:hashmapglobalInit}
\end{table}









%TreeMap
\begin{table}[h]
  \centering
\begin{tabular}{|c|c|c|c|c|}
\hline
Nombre de boîtes & Temps écoulé & CPU & Mémoire totale & Mémoire utilisée\\
\hline
100 & .0074534& 0.05s & 52Mo & 0Mo\\
\hline
1000 & 0.05651 & 0.06s & 52Mo & 4Mo\\
\hline
10000 & 0.14486s & 0.11s & 66Mo & 38Mo\\
\hline
100000 & 2.10083s & 0.38s & 422Mo & 350Mo\\
\hline
1000000 & \multicolumn{4}{|c|}{Out of memory}\\
\hline
\end{tabular}
\caption{Étude de performances avec des TreeMaps globales sans initialisation} 
\label{tab:treemapglobal}
\end{table}

\begin{table}[h]
  \centering
\begin{tabular}{|c|c|c|c|c|}
\hline
Nombre de boîtes & Temps écoulé & CPU & Mémoire totale & Mémoire utilisée\\
\hline
100 & 0.008355 & 0.05s & 52Mo & 7Mo\\
\hline
1000 & 0.03096 & 0.06s & 52Mo & 6Mo\\
\hline
10000 & 0.181671s & 0.13s & 89Mo & 32Mo\\
\hline
100000 & 2.130028 & 0.30s & 481Mo & 318Mo\\
\hline
1000000 & \multicolumn{4}{|c|}{Out of memory}\\
\hline
\end{tabular}
\caption{Étude de performances avec des TreeMaps globales avec initialisation}
\label{tab:treemapglobalInit}
\end{table}



%TreeMap a partir de Hashmap
\begin{table}[h]
  \centering
\begin{tabular}{|c|c|c|c|c|}
\hline
Nombre de boîtes & Temps écoulé & CPU & Mémoire totale & Mémoire utilisée\\
\hline
100 & 0.0085 & 0.05s & 52Mo & 0Mo\\
\hline
1000 & 0.05201 & 0.06s & 52Mo & 4Mo\\
\hline
10000 & 0.126349s & 0.11s & 66Mo & 38Mo\\
\hline
100000 & 2.07826 & 0.39s & 420Mo & 350Mo\\
\hline
1000000 & \multicolumn{4}{|c|}{Out of memory}\\
\hline
\end{tabular}
\caption{Étude de performances avec des TreeMaps construites à partir de HashMaps globales sans initialisation} 
\label{tab:treehashmapglobal}
\end{table}

\begin{table}[h]
  \centering
\begin{tabular}{|c|c|c|c|c|}
\hline
Nombre de boîtes & Temps écoulé & CPU & Mémoire totale & Mémoire utilisée\\
\hline
100 & 0.008646 & 0.05s & 52Mo & 0Mo\\
\hline
1000 & 0.0.04127 & 0.07s & 52Mo & 6Mo\\
\hline
10000 & 0.18813s & 0.10s & 89Mo & 32Mo\\
\hline
100000 & 2.12998599 & 0.29s & 482Mo & 317Mo\\
\hline
1000000 & \multicolumn{4}{|c|}{Out of memory}\\
\hline
\end{tabular}
\caption{Étude de performances avec des TreeMaps construites à partir de HashMaps globales avec initialisation} 
\label{tab:treehashmapglobalInit}
\end{table}
