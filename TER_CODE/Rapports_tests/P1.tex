\chapter{Accès au élément d'une boîte}
Dans cette partie nous allons effectuer des tests d'accès au éléments d'une boîte. Le nombre de coordonnées et de caractéristiques sont les même que dans la partie \ref{chap:Chargement}. De même le remplissage de chaque type de caractéristique demeure de un tiers pour chaque. Nous effectuerons d'abord des tests d'accès au caractéristiques puis des tests d'accès au coordonnées. 

Il est aussi important de préciser que, pour tester, nous avons créer dix-mille boîtes et que pour accèder à une caractéristique, nous accèdons d'abord à une boîte aléatoirement. Les boîtes sont stockés dans un tableau à accès direct par indice. Même s'il est vrai que cette méthode introduit une constante dans le calcul de la complexité, elle permet de ne pas toujours accèder à la même information et donc de ne pas permettre un accès plus rapide qui ne serait pas réaliste.

\section{Tests d'accès au caractéristiques}
Dans tous les cas nous demanderons une caractéristiques en générant aléatoirement un identifiant.

\subsection{Maps internes}
Résultats tableau \ref{tab:accesHM}
