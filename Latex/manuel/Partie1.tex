\chapter{Installation et configuration}

\section{Configuration necessaire}
\label{sec:conf}
\danger : LoCD est un outil open source dédié uniquement aux systèmes d'exploitation linux. Il n'existe pas encore de version pour Windows et MAC OS. L'installation recquière des connaissances dans la manipulation de commandes shell. L'ouvrage suivant est une référence dans ce domaine : \cite{Nutshell}    

\section{Installation}
\label{sec:install}
Rendez vous sur \url{http://www.LoCD.org}. La rubrique «Download» vous proposera une archive de type tar.gz pour différentes distributions (solaris, Linux 32 Bit, Linux 64 Bit, \dots ). Le téléchargement terminé, décompressez l'archive dans le dossier où vous désirez installer LoCD. Placez vous dans ce dossier et tapez la commande \verb+make+. LoCD est maintenant installé sur votre ordinateur \smiley !!  


