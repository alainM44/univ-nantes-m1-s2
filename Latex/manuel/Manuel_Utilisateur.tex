\documentclass[12pt,a4paper,utf8x,titlepage]{report}
\usepackage [frenchb]{babel}
\usepackage[utf8]{inputenc}  
\usepackage[T1]{fontenc} 
% Pour pouvoir utiliser 
\usepackage{ucs}
\usepackage{textcomp}
\usepackage{graphicx}
\usepackage{keystroke}
\usepackage{amssymb}
\usepackage{amsmath}
%\usepackage{pifont}
\usepackage{url} % Pour avoir de belles url
\usepackage{geometry}
\usepackage{hyperref}


\usepackage {listings}% Pour mettre du code source
\lstset{language=sh}

% Pour pouvoir passer en paysage
\usepackage{lscape}

% Pour pouvoir faire plusieurs colonnes
\usepackage {multicol}

\usepackage{makeidx}% Pour crééer un index
\usepackage{graphicx}
\usepackage{fourier-orns} %logo comme \danger
\usepackage[cc]{titlepic} %rajouter le logo LoCD dans la page de garde
\usepackage{tocbibind}
\usepackage{wasysym} %emoticones

\usepackage{glossaries} % Créer un glossaire

\hypersetup{
  backref=true,
  %permet d'ajouter des liens dans...
  pagebackref=true,%...les bibliographies
  hyperindex=true, %ajoute des liens dans les index.
  colorlinks=true, %colorise les liens
  breaklinks=true, %permet le retour à la ligne dans les liens trop longs
  urlcolor= blue, %couleur des hyperliens
  linkcolor= blue, %couleur des liens internes
  bookmarks=true, %créé des signets pour Acrobat
  bookmarksopen=true,
  %si les signets Acrobat sont créés,
  %les afficher complètement.
  pdftitle={Manuel Utilisateur pour LoCD}, %informations apparaissant dans
  pdfauthor={MARGUERITE Alain\\ RINCE Romain},
  %dans les informations du document
  pdfsubject={LoCD}
  %sous Acrobat.
}
%Entête pied de page
%Définition des entêtes : 
\usepackage{fancyhdr}
\pagestyle{fancy}

\renewcommand{\chaptermark}[1]{\markboth{#1}{}} % Pour avoir le nom du chapitre et de la numérotation.
\renewcommand{\sectionmark}[1]{\markright{\thesection\ #1}}
\fancyhf{} \fancyhead[LE,RO]{\bfseries\thepage}
\fancyhead[LO]{\bfseries\rightmark}
\fancyhead[RE]{\bfseries\leftmark}
\renewcommand{\headrulewidth}{0.5pt}% Pour le trait horizontal.
\addtolength{\headheight}{0.5pt}
\renewcommand{\footrulewidth}{0pt}
\fancypagestyle{plain}{ \fancyhead{}
\renewcommand{\headrulewidth}{0pt}} 


% Définition des pieds de pages : on veut un trait horizontal avec le numéro de page. 
\fancyfoot[C]{\thepage}%numéro de page
\renewcommand{\footrulewidth}{0.5pt} %trait horizontal pied de page


\makeindex


%%%% debut macro pour enlever le nom chapitre %%%%
\makeatletter
\def\@makechapterhead#1{%
  \vspace*{50\p@}%
          {\parindent \z@ \raggedright \normalfont
            \interlinepenalty\@M
            \ifnum \c@secnumdepth >\m@ne
            \Huge\bfseries \thechapter\quad
            \fi
            \Huge \bfseries #1\par\nobreak
            \vskip 40\p@
}}

\def\@makeschapterhead#1{%
  \vspace*{50\p@}%
          {\parindent \z@ \raggedright
            \normalfont
            \interlinepenalty\@M
            \Huge \bfseries  #1\par\nobreak
            \vskip 40\p@
}}
\makeatother
%%%% fin macro %%%%



%Couverture 
\titlepic{\includegraphics[scale=0.30]{img/logo}}
\title{Manuel Utilisateur pour Logiciel de Création de Diagrammes}
\author{MARGUERITE Alain\\ RINCE Romain}
\date{\today}


%\storeglosentry{hist}{name=histoire,description=aventure}
%\makeglossary
\newglossaryentry{titre}{name=TITLE,description={Paramètre à fournir dans le fichier d'entrée pour donner un titre à votre diagramme.}}
\newglossaryentry{sous titre}{name=SUBTITLE,description={Paramètre à fournir dans le fichier d'entrée pour donner un sous titre à votre diagramme.}}
\newglossaryentry{note}{name=NOTE,description={Paramètre à fournir dans le fichier d'entrée pour fournir une note.}}
\newglossaryentry{metad}{name=Meta données,description={Informations à fournir en plus des données statistiques en elles mêmes (titre, note, \dots)}}
\newglossaryentry{unix}{name=Unix,description={Le système Unix est un système d'exploitation multi-utilisateurs, multi-tâches, ce qui signifie qu'il permet à un ordinateur mono ou multi-processeurs de faire exécuter simultanément plusieurs programmes par un ou plusieurs utilisateurs. }}
\begin{document}
\makeglossaries

\maketitle




\clearpage

\tableofcontents

\clearpage

% Pour avoir un interligne de 1,5

\chapter{Introduduction et objectifs}\label{chap:fichDonnees}
\section{Avis au lecteur} 
Ce manuel est destiné à un public désirant utiliser le logiciel LoCD. C'est à dire depuis son installation jusqu'à'à la génération du fichier au format pdf contenant le diagramme désiré. Si une partie est consacrée à la mise en forme de ce fichier de données (\ref{chap:fichDonnees}), nécessaire au fonctionnement de LoCD, ce manuel n'a pas pour objectif d'enseigner les methodes de calculs de ces données statistiques \cite{stat}. %%rajouter biblio 
(chapitre \ref{chap:UseGraph}) 
\section{Présentation du logiciel de création de diagrammes}
LoCD permet la création automatique de diagrammes, histogrammes ou nuages de points à partir d'un fichier de données statistiques.  L'outil peut être utilisé de deux manières différentes : en ligne de commande ou par le biais de son interface graphique. Ces deux methodes seront détaillées dans ce manuel. 

\chapter{Votre premier diagramme}
\section{Introduction}
Ce chapitre va vous permettre de réaliser une vote premier diagramme avec LoCD en moins de 5 minutes  \smiley ! Voici le résultat que vous obtiendrez au terme : 
\begin{figure}[htbp]
  \centering
  \includegraphics[scale=0.60]{img/diagrammenuages}
  \caption{Nuages de points avec toutes les méta données possibles renseignées}
  \label{fig:dnuages}
\end{figure}

\section{1\up{ère} \'Etape : Génération du fichier d'entrée}
Ouvrez un éditeur de texte de votre choix. Saisissez les lignes suivantes (utilisez le copier/coller pour gagner du temps).
\begin{verbatim}
  >TITLE: Les plus grands pays du monde pays (~2010)
  >SUBTITLE: En km²
  >Note: La France n'est que 42ème

  Russie      Canada 	   États-Unis    Chine 	    Brésil 
  17 098 242  9 984 670  9 629 091  	9 596 961   8 514 877 km2 	
\end{verbatim}
Enregistrez le fichier : \verb+mon_premier_diagramme.txt+
\section{2\up{ème} \'Etape : Utilisation de LoCD}
Il ne vous reste plus qu'a lancer la commande suivante :
  \begin{figure}[htbp]
    \centering
    \includegraphics[scale=0.40]{img/ecommandes}
    \caption{Exemple d'utilisation de LoCD en ligne de commandes.}
    \label{fig:ecommandes}
  \end{figure} 
  
Et voilà avez crée votre premier diagramme avec LoCD ! Si vous avez rencontrez des difficultés au cours de ce chapitre, vous pouvez vous référer aux différentes parties de ne manuel qui détaille chaque étapes en détail.

\chapter{Installation}
%\label{chap:install}
Lorem ipsum dolor sit amet, consectetur adipiscing elit. Mauris eu dapibus magna. Cras vel elit vel mauris bibendum pulvinar. Lorem ipsum dolor sit amet, consectetur adipiscing elit. Vivamus posuere velit eget mauris volutpat pellentesque. Integer condimentum magna porta enim aliquet fringilla. Lorem ipsum dolor sit amet, consectetur adipiscing elit. Fusce in ante dolor, vel posuere ipsum.

Donec eu augue quam. Pellentesque blandit elementum tellus non feugiat. Donec volutpat lectus elit. Pellentesque imperdiet dui vitae ligula vulputate sit amet congue urna laoreet. Ut nisl ligula, aliquam eu pretium sed, tincidunt et nunc. Pellentesque lacinia venenatis ligula in lobortis. Aliquam lorem lorem, iaculis non lacinia eget, ultricies non dui. Donec ultrices vehicula augue, ut pellentesque massa imperdiet ac. Maecenas feugiat, massa id posuere vestibulum, lectus risus pulvinar metus, ut fermentum neque mauris eu est. Sed posuere venenatis quam sed volutpat. Quisque pellentesque sem ac nulla consequat sagittis. Praesent elementum dolor eget nisi lacinia eget facilisis nisi bibendum. Nulla in urna nisi. Curabitur vitae nisl augue, eget blandit magna. 
%\section{Configuration necessaire)
Lorem ipsum dolor sit amet, consectetur adipiscing elit. Mauris eu dapibus magna. Cras vel elit vel mauris bibendum pulvinar. Lorem ipsum dolor sit amet, consectetur adipiscing elit. Vivamus posuere velit eget mauris volutpat pellentesque. Integer condimentum magna porta enim aliquet fringilla. Lorem ipsum dolor sit amet, consectetur adipiscing elit. Fusce in ante dolor, vel posuere ipsum.

Donec eu augue quam. Pellentesque blandit elementum tellus non feugiat. Donec volutpat lectus elit. Pellentesque imperdiet dui vitae ligula vulputate sit amet congue urna laoreet. Ut nisl ligula, aliquam eu pretium sed, tincidunt et nunc. Pellentesque lacinia venenatis ligula in lobortis. Aliquam lorem lorem, iaculis non lacinia eget, ultricies non dui. Donec ultrices vehicula augue, ut pellentesque massa imperdiet ac. Maecenas feugiat, massa id posuere vestibulum, lectus risus pulvinar metus, ut fermentum neque mauris eu est. Sed posuere venenatis quam sed volutpat. Quisque pellentesque sem ac nulla consequat sagittis. Praesent elementum dolor eget nisi lacinia eget facilisis nisi bibendum. Nulla in urna nisi. Curabitur vitae nisl augue, eget blandit magna. 








\chapter{Utilisation du logiciel grâce à l'interface}
\section{Lancement du logiciel}
Une fois le logiciel installé (Voir le chapitre \ref{chap:install} sur l'installation du logiciel)




\chapter{Fichier d'entrée}\label{chap:fichDonnees}


\chapter{Fonctionalités}
Nous détaillerons dans cette partie les différentes fonctionalités que propose l'outil. Des exemples illustrés et des \dots  
\section{Histogrammes}
Type de diagramme répendu, l'histogramme fait partie des diagrammes que LoCD peut générer. L'exmple ci-dessus illustre un résultat basique avec la configure par défaut de LoCD soit : 
\begin{itemize}
\item
  Une unique couleur : bleu
\item
  Absence de titre, sous titre et notes
\item
  Représentation 2D
\end{itemize}

\begin{figure}[htbp]
  \centering
  \includegraphics[scale=0.40]{img/diagrammebaton}
  \caption{Histogramme avec les paramètres par défaults}
  \label{fig:dbatons}
\end{figure}
Pour changer cette configuration par défault, se référer au chapitre configuration% références !!!!!

\section{Diagrammes circulaires}
Appelés un diagramme « en camembert » (pie-chart en anglais pour sa forme en tarte), ce type de diagramme utilisé en statistiques. 
\begin{figure}[htbp]
  \centering
  \includegraphics[scale=0.60]{img/diagrammecirculaire}
  \caption{Exemple avec un titre et un sous titre fournis dans les métas données.}
  \label{fig:dcirculaire}
\end{figure}
Sur cet exemple, plusieurs paramètres par défaut ont été modifiés. Les couleurs notamment. Pour apprendre comment effectuer un tel réglage se référer à la partie suivante : ~\ref{subsec:couleurs}
\clearpage

\section{Nuages de points}
Diagramme fréquement utilisée dans la représentation dans les séries statistiques à deux variables. LoCD permet de généger ce type de diagramme. L'exemple présenté dans la figure suivante, rassemble la plupart des fonctionalité que propose LoCD.
\begin{figure}[htbp]
  \centering
  \centering
  \includegraphics[scale=0.60]{img/diagrammenuages}
  \caption{Nuages de points avec toutes les méta données possibles renseignées}
  \label{fig:dnuages}
\end{figure}  

Les configuration en mode ligne de commandes sont détaillée dans ici : \ref{subsec:excom}.

\section{Utilisation par un terminal}
\sectionmark{Terminal}
\label{chap:useterm}

\renewcommand{\labelitemi}{$\bullet$} %changer les puces pour cette page

GT peut être utilisé uniquement en ligne de commande. Cette partie demandes des connaissances pré requises sur les commandes unix. En effet seul les fonctionnalités de l'outil seront explicitées. Le mécanisme des options est similaire à toute autres commandes unix. Pour plus de d'information sur les système \gls{unix}
, nous recommandons l'ouvrage suivant : \cite{linux}.

%%%%%%%%%%%%%%%%%%%%%%%%%%%%%%%%%%%%%%%%%%%%%%%%%%%%%%%%%%%%%%%%%%%%%%%%%

\subsection{Utilisation basique}
\label{sec:usebas}
La simple commande suivante exécutera l'outil GC \verb+java -jar GT+  vous donne la possibilité d'entez un nom de fichier. Vous avez alors deux possibilités : 
\begin{itemize}
\item
Utiliser une démonstration des algorithmes à partir de tâches prédéfinies. Vous devrez par la suite choisir l'ordonnancement que vous souhaitez visualiser. 
\item
Générer vos propres tâches. Suivez les instructions pour la génération de celles ci.
\end{itemize}

%%%%%%%%%%%%%%%%%%%%%%%%%%%%%%%%%%%%%%%%%%%%%%%%%%%%%%%%%%%%%%%%%%%%%%%%%

\subsection{Génération manuelle des tâches}
Il vous sera demandé successivement si vous souhaitez une génération automatique ou des nom des tâches périodiques. Dans le cas d'une génération automatique il est nécessaire de renseigner le pourcentage d'utilisation maximale du processeur. Pour les tâches apériodiques il est aussi nécessaire de fournir leur taux d'occupation du processeur (U\_a).
Dans le cas d'une génération entièrement manuelle, il est demandé de fournir chaque caractéristiques de la tâche. Pour plus de détails sur chaque attribut à fournir nous vous invitons à vous référer au document suivant : \cite{SD}
 

\chapter{Copyright}
	\paragraph{}
    Ce programme est un logiciel libre : vous pouvez le redistribuer ou
    le modifier selon les termes de la GNU General Public Licence tels
    que publiés par la Free Software Foundation : à votre choix, soit la
    version 3 de la licence, soit une version ultérieure quelle qu'elle
    soit.
	\paragraph{}
    Ce programme est distribué dans l'espoir qu'il sera utile, mais SANS
    AUCUNE GARANTIE ; sans même la garantie implicite de QUALITÉ
    MARCHANDE ou D'ADÉQUATION À UNE UTILISATION PARTICULIÈRE. Pour
    plus de détails, reportez-vous à la GNU General Public License.
	\paragraph	{}
    Vous devez avoir reçu une copie de la GNU General Public License
    avec ce programme. Si ce n'est pas le cas, consultez
 \cite{GNU}   
  \begin{figure}[htbp]
    \centering
    \includegraphics{img/gpl}
  \end{figure}  


% Pour finir l'interligne de 1,5
\printglossary

\listoffigures
\printindex



\appendix
%\bibliographystyle{AB_bib}
\bibliographystyle{alpha}
\bibliography{biblio.bib}

\end{document}

%Note : Installation package, décompressez les package dans un dossier. Changer la var set TEXINPUTS=/path/files
