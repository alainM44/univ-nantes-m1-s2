\documentclass[12pt,a4paper,utf8x]{report}
\usepackage [frenchb]{babel}
\usepackage[utf8]{inputenc}  
\usepackage[T1]{fontenc} 
% Pour pouvoir utiliser 
\usepackage{ucs}

\usepackage{textcomp}
\usepackage{graphicx}
\usepackage{keystroke}
\usepackage{amssymb}
\usepackage{amsmath}
%\usepackage{pifont}

\usepackage{url} % Pour avoir de belles url
\usepackage {geometry}
\usepackage{hyperref}
%[linktocpage]
% Pour mettre du code source
\usepackage {listings}
% Pour pouvoir passer en paysage
\usepackage{lscape}

% Pour pouvoir faire plusieurs colonnes
\usepackage {multicol}
% POur crééer un index
\usepackage{makeidx}
\usepackage{graphicx}
\usepackage{tocbibind}
\hypersetup{
backref=true,
%permet d'ajouter des liens dans...
pagebackref=true,%...les bibliographies
hyperindex=true, %ajoute des liens dans les index.
colorlinks=true, %colorise les liens
breaklinks=true, %permet le retour à la ligne dans les liens trop longs
urlcolor= blue, %couleur des hyperliens
linkcolor= blue, %couleur des liens internes
bookmarks=true, %créé des signets pour Acrobat
bookmarksopen=true,
%si les signets Acrobat sont créés,
%les afficher complètement.
pdftitle={Manuel Utilisateur pour LoCD}, %informations apparaissant dans
pdfauthor={MARGUERITE Alain\\ RINCE Romain},
%dans les informations du document
pdfsubject={LoCD}
%sous Acrobat.
}
%pied de page
\usepackage{fancyhdr}
\pagestyle{fancy}
\fancyhf{}
\fancypagestyle{plain}{
\fancyfoot[C]{\thepage}
\fancyhead[C]{\leftmark} 
 %\renewcommand{\headrulewidth}{0pt} % ...et le filet
}
%\renewcommand{\headrulewidth}{0.5pt}  %trait horizontal en tête
\renewcommand{\footrulewidth}{0.5pt} %trait horizontal pied de page

%\renewcommand{\chaptermark}[1]{\markboth{#1}{}}
\renewcommand{\sectionmark}[1]{\markright{#1}}

%\renewcommand{\footrulewidth}{0.5pt}
\makeindex


%%%% debut macro pour enlever le nom chapitre %%%%
\makeatletter
\def\@makechapterhead#1{%
  \vspace*{50\p@}%
  {\parindent \z@ \raggedright \normalfont
    \interlinepenalty\@M
    \ifnum \c@secnumdepth >\m@ne
        \Huge\bfseries \thechapter\quad
    \fi
    \Huge \bfseries #1\par\nobreak
    \vskip 40\p@
  }}

\def\@makeschapterhead#1{%
  \vspace*{50\p@}%
  {\parindent \z@ \raggedright
    \normalfont
    \interlinepenalty\@M
    \Huge \bfseries  #1\par\nobreak
    \vskip 40\p@
  }}
\makeatother
%%%% fin macro %%%%

%Couverture 


\title
{Manuel Utilisateur pour LoCD}



\author{MARGUERITE Alain\\ RINCE Romain}

\date{Université de Nantes \\ 2 rue de la Houssinière, BP92208, F-44322 Nantes cedex 03, FRANCE}

\begin{document}

\maketitle


\clearpage

\tableofcontents
\clearpage

% Pour avoir un interligne de 1,5
%\chapter{Introduction}
%Dans le cadre du module Initiation à la recherche, 
L'objectif de ce document est de décrire les notions essentielles à retenir en ce début de projet d'initiation à la recherche. Ces notions font parties d'un même sujet d'étude, au coeurs des travaux de l'équipe OPTI. Elles concernent le sujet des contraintes et des intervalles. Aussi des notions d'optimisation seront abordées.

\section{Méthodes de calculs}
Les méthodes de calcul permettant de résoudre des CSP (Constraint Satisfaction Solveur) ou GCSP (Geometric Constraint Satisfaction Solveur) peuvent être divisées en deux catégories les methodes formelles et les methodes numériques. Après avoir brièvement rappellé les grandes lignes et les outils qui dérivent de ces méthodes..

\subsection{Méthodes formelles}
Elle permettent un une résolution d'un système d'equations. Les variables peuvent être des inconnues complexes réelles ou rationelles. Les approches les plus classiques des méthodes formelles utilisent des théories, telles que les idéaux polynomiaux pour les bases de Gröbner, ou la théorie des déterminant pour la méthode du résultant.  Leurs points forts résident dans leur capacité d'éffectuer des résolutions exactes et complète d'un système d'équation. Les méthodes formelles présentent cependant l'inconveinient de mettre en oeuvre des algorithmes de comlpexité exponentielles.

\subsection{Méthodes numérique}
Capablent de résoudre n'importe quel système d'équation (ou d'égalités) les méthodes numériques présentent cependant un important point faible. Il réside dans les opération d'approximations (d'arrondis)  pouvant entraîner des résultats complétement faux. Cependant ces calculs numériques utilisés  par des méthodes de résolutions par intervalles permettent de contourner ces problèmes. Ainsi nou


blah blahblah blahblah \cite{Jermann} blahblah blahblah blahblah\cite{Goldsztejn} blahblah blah  \cite{Goualard}


dsfsdfsdf\cite{Schichl} dsfsdfsdfsdf \cite{Neumaier}

\chapter{Introduduction et objectifs}\label{chap:fichDonnees}
\section{Avis au lecteur} 
Ce manuel est destiné à un public désirant utiliser le logiciel LoCD. C'est à dire depuis son installation jusqu'à'à la génération du fichier au format pdf contenant le diagramme désiré. Si une partie est consacrée à la mise en forme de ce fichier de données (\ref{chap:fichDonnees}), nécessaire au fonctionnement de LoCD, ce manuel n'a pas pour objectif d'enseigner les methodes de calculs de ces données statistiques \cite{stat}. %%rajouter biblio 
(chapitre \ref{chap:UseGraph}) 
\section{Présentation du logiciel de création de diagrammes}
LoCD permet la création automatique de diagrammes, histogrammes ou nuages de points à partir d'un fichier de données statistiques.  L'outil peut être utilisé de deux manières différentes : en ligne de commande ou par le biais de son interface graphique. Ces deux methodes seront détaillées dans ce manuel. 

\chapter{Installation}
%\label{chap:install}
Lorem ipsum dolor sit amet, consectetur adipiscing elit. Mauris eu dapibus magna. Cras vel elit vel mauris bibendum pulvinar. Lorem ipsum dolor sit amet, consectetur adipiscing elit. Vivamus posuere velit eget mauris volutpat pellentesque. Integer condimentum magna porta enim aliquet fringilla. Lorem ipsum dolor sit amet, consectetur adipiscing elit. Fusce in ante dolor, vel posuere ipsum.

Donec eu augue quam. Pellentesque blandit elementum tellus non feugiat. Donec volutpat lectus elit. Pellentesque imperdiet dui vitae ligula vulputate sit amet congue urna laoreet. Ut nisl ligula, aliquam eu pretium sed, tincidunt et nunc. Pellentesque lacinia venenatis ligula in lobortis. Aliquam lorem lorem, iaculis non lacinia eget, ultricies non dui. Donec ultrices vehicula augue, ut pellentesque massa imperdiet ac. Maecenas feugiat, massa id posuere vestibulum, lectus risus pulvinar metus, ut fermentum neque mauris eu est. Sed posuere venenatis quam sed volutpat. Quisque pellentesque sem ac nulla consequat sagittis. Praesent elementum dolor eget nisi lacinia eget facilisis nisi bibendum. Nulla in urna nisi. Curabitur vitae nisl augue, eget blandit magna. 
%\section{Configuration necessaire)
Lorem ipsum dolor sit amet, consectetur adipiscing elit. Mauris eu dapibus magna. Cras vel elit vel mauris bibendum pulvinar. Lorem ipsum dolor sit amet, consectetur adipiscing elit. Vivamus posuere velit eget mauris volutpat pellentesque. Integer condimentum magna porta enim aliquet fringilla. Lorem ipsum dolor sit amet, consectetur adipiscing elit. Fusce in ante dolor, vel posuere ipsum.

Donec eu augue quam. Pellentesque blandit elementum tellus non feugiat. Donec volutpat lectus elit. Pellentesque imperdiet dui vitae ligula vulputate sit amet congue urna laoreet. Ut nisl ligula, aliquam eu pretium sed, tincidunt et nunc. Pellentesque lacinia venenatis ligula in lobortis. Aliquam lorem lorem, iaculis non lacinia eget, ultricies non dui. Donec ultrices vehicula augue, ut pellentesque massa imperdiet ac. Maecenas feugiat, massa id posuere vestibulum, lectus risus pulvinar metus, ut fermentum neque mauris eu est. Sed posuere venenatis quam sed volutpat. Quisque pellentesque sem ac nulla consequat sagittis. Praesent elementum dolor eget nisi lacinia eget facilisis nisi bibendum. Nulla in urna nisi. Curabitur vitae nisl augue, eget blandit magna. 








\chapter{Utilisation du logiciel grâce à l'interface}
\section{Lancement du logiciel}
Une fois le logiciel installé (Voir le chapitre \ref{chap:install} sur l'installation du logiciel)




\chapter{Fichier d'entrée}\label{chap:fichDonnees}


\chapter{Fonctionalités}
Nous détaillerons dans cette partie les différentes fonctionalités que propose l'outil. Des exemples illustrés et des \dots  
\section{Histogrammes}
Type de diagramme répendu, l'histogramme fait partie des diagrammes que LoCD peut générer. L'exmple ci-dessus illustre un résultat basique avec la configure par défaut de LoCD soit : 
\begin{itemize}
\item
  Une unique couleur : bleu
\item
  Absence de titre, sous titre et notes
\item
  Représentation 2D
\end{itemize}

\begin{figure}[htbp]
  \centering
  \includegraphics[scale=0.40]{img/diagrammebaton}
  \caption{Histogramme avec les paramètres par défaults}
  \label{fig:dbatons}
\end{figure}
Pour changer cette configuration par défault, se référer au chapitre configuration% références !!!!!

\section{Diagrammes circulaires}
Appelés un diagramme « en camembert » (pie-chart en anglais pour sa forme en tarte), ce type de diagramme utilisé en statistiques. 
\begin{figure}[htbp]
  \centering
  \includegraphics[scale=0.60]{img/diagrammecirculaire}
  \caption{Exemple avec un titre et un sous titre fournis dans les métas données.}
  \label{fig:dcirculaire}
\end{figure}
Sur cet exemple, plusieurs paramètres par défaut ont été modifiés. Les couleurs notamment. Pour apprendre comment effectuer un tel réglage se référer à la partie suivante : ~\ref{subsec:couleurs}
\clearpage

\section{Nuages de points}
Diagramme fréquement utilisée dans la représentation dans les séries statistiques à deux variables. LoCD permet de généger ce type de diagramme. L'exemple présenté dans la figure suivante, rassemble la plupart des fonctionalité que propose LoCD.
\begin{figure}[htbp]
  \centering
  \centering
  \includegraphics[scale=0.60]{img/diagrammenuages}
  \caption{Nuages de points avec toutes les méta données possibles renseignées}
  \label{fig:dnuages}
\end{figure}  

Les configuration en mode ligne de commandes sont détaillée dans ici : \ref{subsec:excom}.

\section{Utilisation par un terminal}
\sectionmark{Terminal}
\label{chap:useterm}

\renewcommand{\labelitemi}{$\bullet$} %changer les puces pour cette page

GT peut être utilisé uniquement en ligne de commande. Cette partie demandes des connaissances pré requises sur les commandes unix. En effet seul les fonctionnalités de l'outil seront explicitées. Le mécanisme des options est similaire à toute autres commandes unix. Pour plus de d'information sur les système \gls{unix}
, nous recommandons l'ouvrage suivant : \cite{linux}.

%%%%%%%%%%%%%%%%%%%%%%%%%%%%%%%%%%%%%%%%%%%%%%%%%%%%%%%%%%%%%%%%%%%%%%%%%

\subsection{Utilisation basique}
\label{sec:usebas}
La simple commande suivante exécutera l'outil GC \verb+java -jar GT+  vous donne la possibilité d'entez un nom de fichier. Vous avez alors deux possibilités : 
\begin{itemize}
\item
Utiliser une démonstration des algorithmes à partir de tâches prédéfinies. Vous devrez par la suite choisir l'ordonnancement que vous souhaitez visualiser. 
\item
Générer vos propres tâches. Suivez les instructions pour la génération de celles ci.
\end{itemize}

%%%%%%%%%%%%%%%%%%%%%%%%%%%%%%%%%%%%%%%%%%%%%%%%%%%%%%%%%%%%%%%%%%%%%%%%%

\subsection{Génération manuelle des tâches}
Il vous sera demandé successivement si vous souhaitez une génération automatique ou des nom des tâches périodiques. Dans le cas d'une génération automatique il est nécessaire de renseigner le pourcentage d'utilisation maximale du processeur. Pour les tâches apériodiques il est aussi nécessaire de fournir leur taux d'occupation du processeur (U\_a).
Dans le cas d'une génération entièrement manuelle, il est demandé de fournir chaque caractéristiques de la tâche. Pour plus de détails sur chaque attribut à fournir nous vous invitons à vous référer au document suivant : \cite{SD}
 


% Pour finir l'interligne de 1,5
\listoffigures
\printindex

\appendix


\end{document}
