\Softname{} is a modeling language for numerical constraint solving.  In
this section, we will describe how problems are modeled using \Softname{} and
how models are solved.

%------------------------------------------------------------------------------------------
\unnumberedpar{Constraint Programming}

\addcontentsline{toc}{subsubsection}{Constraint Programming}

\index{Constraint Programming}
\index{constraint}
\index{variable}
\index{feasible solution}
\index{consistent value}
\index{inconsistent value}
\index{search}
\index{solver}
\index{variable!domain}
\index{domain}
\index{consistency technique}
\index{model}
\index{reduction}

A \emph{constraint} is a relation between the unknowns of a problem,
\emph{i.e.}, the \emph{variables}. For instance, in geometric modeling, one wants
to determine a point $(x,y)$ in $\R^2$, which distance to the origin is
greater than $2$. The corresponding mathematical formula,
%
$$
  \sqrt{x^2 + y^2} \geq 2,
$$
%
is just what we call a constraint. The problem is to find values of
the variables that satisfy the constraint, or more generally all the
constraints of a model. The point $(1,2)$ is a feasible solution since
$\sqrt{1^2 + 2^2} \geq 2$. This value is declared
\emph{consistent} with respect to the constraint.

In Constraint Programming, the search of solutions, which is done by
\emph{solvers}, consists in eliminating inconsistent values.
Initially, a \emph{domain} is associated to each variable,
\emph{i.e.}, the set of possible values \emph{a priori}.
The solvers implement domain reductions by means of consistency
techniques and domain splitting in order to separate the solutions.

Constraint programming allows the user to design models with
constraints in a very natural way, as the scientist does. The
constraint programming system is responsible for the search of
solutions.


%------------------------------------------------------------------------------------------
\unnumberedpar{Numerical Constraints}

\addcontentsline{toc}{subsubsection}{Numerical Constraints}

\Softname{} is able to solve \emph{numerical constraints}, \emph{i.e.},
nonlinear equations or inequations over the real numbers.  Each
constraint is defined by an analytic expression
%
$$
  f(x_1,\dots,x_n)\diamond 0
$$
%
where $\diamond\in\{=,\leq,\geq\}$ and $f:\ \R^n\to\R$ is a function.

\index{interval}
\index{IEEE floating-point number}

Each domain is represented by a \emph{closed interval} whose bounds
belong to the set of IEEE floating-point numbers
(see~\cite{IEEE754}). Four kinds of intervals are manipulated:
%
$$
\begin{array}{rcl}
  \Ibounds{r_1}{r_2} & = & \{r\in\R\mid r_1\leq r\leq r_2\}\\
  \IboundsCO{r_1}{+\infty} & = & \{r\in\R\mid r_1\leq r\}\\
  \IboundsOC{-\infty}{r_2} & = & \{r\in\R\mid r\leq r_2\}\\
  \IboundsOO{-\infty}{+\infty} & = & \R
\end{array}
$$
%

\index{model}
\index{CSP}

The notion of model used in \Softname{} is quite simple: a model is a
\emph{constraint satisfaction problem} (CSP) that is
composed of:
%
\begin{itemize}
\item A set of real-valued variables, \emph{e.g.}, $\{x_1,\dots,x_n\}$;

\item A set of interval domains, \emph{e.g.}, $\{\Ibold{x_1},\dots,\Ibold{x_n}\}$;

\item A set of numerical constraints, \emph{e.g.}, $\{c_1,\dots,c_m\}$
   over the given set of variables.
\end{itemize}

The problem is to find in the initial \emph{box}
$\Ibold{x_1}\times\cdots\times\Ibold{x_n}$ all the consistent values
with respect to all constraints.

\index{box}
\index{variable}
\index{domain}
\index{constraint}

%------------------------------------------------------------------------------------------
%------------------------------------------------------------------------------------------
\subsection{Modeling with Constraints}

In this section, we present the basic features of \Softname{}.

%------------------------------------------------------------------------------------------
\unnumberedpar{Example: Intersection of Circles}

\addcontentsline{toc}{subsubsection}{Example: Intersection of Circles}

\noindent Consider the following problem: given a real number $d\geq 0$
and a circle with center $(x_0,y_0)$ and radius $r$, find all the
points on the circle whose distance to the origin is $d$. As shown in
Fig.~\ref{fig:overview:circle}, this is a problem of intersection of circles.

\begin{figure}[htbp]
\hspace*{54mm}\includegraphics[bb=130 610 420 750]{figures/circle-problem.ps}
\caption{Intersection of Circles.}
\label{fig:overview:circle}
\end{figure}

In this problem, we may identify the following data:
%
\begin{itemize}
\item $4$ constants: $d$, $x_0$, $y_0$, $r$. Depending on their values,
the problem has $0$, $1$ or $2$ solutions.

\item $2$ variables: the coordinates $(x,y)$ of the point.
  Since these  variables lie \emph{a priori} in $\R$,
  the variable domains are defined as the set $\R$.

\item $2$ constraints:
  \begin{itemize}
  \item  $(x,y)$ is required to be on the given circle, \emph{i.e.},
  %
  $$
    r^2 = (x-x_0)^2 + (y-y_0)^2.
  $$
  %

  \item the distance of $(x,y)$ to the origin is $d$, \emph{i.e.},
  %
  $$
    d^2 = x^2 + y^2.
  $$
  %
  \end{itemize}
\end{itemize}

Given a particular set of constants, the complete program follows.
%
\begin{envcode}
\begin{verbatim}
Constants
  x0 = 2, y0 = 1 ,               /* center of the given circle */
  r  = 1.25      ,               /* radius */
  d  = 2.75      ;               /* distance to origin */

Variables
  x in ]-oo, +oo[ ,              /* definition of variables */
  y in ]-oo, +oo[ ;              /* and domains (the set R) */

Constraints
  r^2 = (x-x0)^2 + (y-y0)^2 ,    /* the point (x,y) is on the circle */
  d^2 = x^2 + y^2           ;    /* the distance from the origin is d */
\end{verbatim}
\end{envcode}

\medskip

\noindent The output is:
%
\begin{envresult}
\begin{verbatim}
OUTER BOX 1
  x in [2.75 , 2.75]
  y in [-2.119516683375301e-16 , +0]

  precision: 4.44e-16, elapsed time: 0 ms

OUTER BOX 2
  x in [1.649999999999999 , 1.650000000000001]
  y in [2.199999999999999 , 2.200000000000001]

  precision: 2.44e-15, elapsed time: 10 ms

END OF SOLVING
  Property:     reliable process (no solution is lost)
  Elapsed time: 10 ms
\end{verbatim}
\end{envresult}

\medskip


The interpretation of the result is the following:
%
\begin{itemize}
\item The solving process is \emph{reliable}: this is the essential property of
\Softname{} (Prop.~\ref{property:overview:complete}), which guarantees
that the union of all the computed boxes contain all the solutions of
the CSP.

\item An \emph{outer box} is a box which may contain some solutions
  of the CSP. Here, the result is the union of two boxes, each solution
  being contained in one box.
\end{itemize}

Suppose now that for another set of constants, the CSP has no solution,
for instance given $x_0=y_0=1$, $r=10$ and $d=1$.
What is the result? In this case, \Softname{} guarantees
that the problem has no solution.
%
\begin{envresult}
\begin{verbatim}
END OF SOLVING
  Property:     no solution in the initial box
  Elapsed time: 0 ms
\end{verbatim}
\end{envresult}

\index{reliability}
\index{satisfiability}

\begin{property}[Reliability]\label{property:overview:complete}
Given a CSP, \Softname{} computes a union of boxes that contains all the
solutions of the CSP.
\end{property}
%

In the literature, this property is often called completeness
property. However we believe that there may be a confusion with
respect to the notion of completeness in Constraint Logic Programming,
where a complete solver is able to decide for the satisfiability of a
CSP (it always answers \textit{no} if the CSP has no solution, and
\textit{yes} otherwise). Actually, we just use the terminology
of the interval framework.

Property~\ref{property:overview:complete} has a very important
corollary that allows one to detect the unsatisfiability of a CSP.

\begin{corollary}[Property~\ref{property:overview:complete}]
Given a CSP, if no box is computed by \Softname{}, then the CSP has no
solution.
\end{corollary}

%\vspace*{-2mm}

%------------------------------------------------------------------------------------------
\unnumberedpar{Constants}

\addcontentsline{toc}{subsubsection}{Constants}

\index{constant}
\index{expression}

A constant is a symbol defined by a numerical value or more generally
a constant expression involving real numbers, operations and already
defined constants.

For instance consider the following piece of program.
%
\begin{envcode}
\begin{verbatim}
Constants  a = 4           ,
           b = 1 - sqrt(a) ,
           c = log(2)      ,
           d = 1.1         ;
\end{verbatim}
\end{envcode}

\Softname{} just computes an interval for each constant. Moreover, it guarantees
that the real number defined by the right-hand expression is contained
in the computed interval,  as follows.
%
\begin{envresult}
\begin{verbatim}
a = 4
b = -1
c in [0.69314718055994506418215905796387 , 0.69314718055994550827136890802649]
d in [1.0999999999999998667732370449812 , 1.1000000000000000888178419700125]
\end{verbatim}
\end{envresult}


\index{constant!predefined}

There are several predefined constants in the language,
each one being prefixed by the symbol \texttt{@}, as shown
in this table:

$$
\begin{array}{|lc|lc|lc|lc|}
\hline
\hbox{constant} & \hbox{value} &
\hbox{constant} & \hbox{value} &
\hbox{constant} & \hbox{value} &
\hbox{constant} & \hbox{value}\\
\hline
\texttt{@pi}        & \pi            &
\texttt{@2\_pi}     & 2\pi           &
\texttt{@3\_pi}     & 3\pi           &
\texttt{@4\_pi}     & 4\pi           \\[2mm]
\texttt{@pi\_2}     & \frac{\pi}{2}  &
\texttt{@3\_pi\_2}  & \frac{3\pi}{2} &
\texttt{@5\_pi\_2}  & \frac{5\pi}{2} &
\texttt{@7\_pi\_2}  & \frac{7\pi}{2} \\[2mm]
\texttt{@inv\_pi}   & \frac{1}{\pi}  &
\texttt{@sqrt2}     & \sqrt{2}       &
\texttt{@e}         & \exp(1)        &
\texttt{@log2}      & \log_e(2)      \\[1mm]
\hline
\end{array}
$$

\medskip

\newpage

%------------------------------------------------------------------------------------------
\unnumberedpar{Variables and Domains}

\addcontentsline{toc}{subsubsection}{Variables and Domains}

\index{variable!integer}
\index{variable!real}
\index{type}

\Softname{} is able to represent two types of variables: integer and real
variables. The default type is real. However, a type may be specified
at the declaration, as follows.
%
\begin{envcode}
\begin{verbatim}
Variables
  real    x in ... ,
  integer n in ... ;
\end{verbatim}
\end{envcode}

There is no specific solver for integer variables and constraints in
\Softname{}. There is just an additionnal constraint for each integer variable
that allows one to reduce the domain bounds to integer values as soon as
possible. Note that \Softname{} is necessarily slower for integer constraints
than specific solvers.

\index{integer constraints}

As for constants, each domain bound can be defined by a constant
expression.
%
\begin{envcode}
\begin{verbatim}
Variables  x in [1.1, 1 + @log2], ...
\end{verbatim}
\end{envcode}

\index{expression}
\index{rounding error}


The result below shows that each expression is evaluated by interval
computations, using the outward rounding mode of floating-point
computations. The left bound corresponds to an evaluation of $1.1$
rounded towards $(-\infty)$ while the right bound corresponds to an
evaluation of $(1 + \log(2))$ rounded towards $(+\infty)$. As a
consequence, \Softname{} guarantees that the input domain
(the one in the user's mind) is included in the computed domain.
%
\begin{envresult}
\begin{verbatim}
x in [1.0999999999999998667732370449812 , 1.6931471805599453972490664455108]
\end{verbatim}
\end{envresult}

\index{infinities}
\index{IEEE floating-point number}

In the IEEE standard, the infinities are completely specified so as to
define infinite projective values for the set of floating-point
numbers. Their use as interval bounds is allowed in
\Softname{}. However, note that rounding errors are non negligible near the infinities,
and it may be difficult to remove these values from domains. Should we
use the infinities or not? Yes, but only if no more information is
known about the variables. For instance, given a variable that
represents a distance between two places on Earth, the domain
$\Ibounds{0}{5e4}$ is preferred to $\R$ since $5e4$ is an upper bound
of Earth's circumference.

In many situations the definition of arrays of variables is required.
This is done as follows.
%
\begin{envcode}
\begin{verbatim}
Variables  x[1..3] in [-1, 1] ;
\end{verbatim}
\end{envcode}
%

This just declares $3$ different variables
\verb#x[1]#, \verb#x[2]#, \verb#x[3]# having the same initial domain.

\index{variable!array}

Finally, there is often a motivation to share complex expressions
between constraints. This can be done by using a new variable lying in
$\R$, defining a new equation between the variable and the expression
and replacing the expression by the variable in the constraints.
However, the value of this variable has generally no sense for the
user.  In \Softname{} there exists a mechanism for hidding such
variables, just adding the \texttt{\$} symbol in the definition of the
variable as follows.
%
\begin{envcode}
\begin{verbatim}
Variables  $z in ]-oo, +oo[, ...     /* hidden variable */
\end{verbatim}
\end{envcode}
%

\index{variable!hidden}

%------------------------------------------------------------------------------------------
\unnumberedpar{Example: Disjunctive Constraints}

\addcontentsline{toc}{subsubsection}{Example: Disjunctive Constraints}

A second problem is considered now: given a point $(x_0,y_0)$ and a
real $d\geq 0$, find all the points $(x,y)$ such that $x$ is an
integer in $[-5,7]$, $d$ is the distance between $(x,y)$ and
$(x_0,y_0)$, and either $y\leq 0$ or $y\geq x$.

The disjunction of constraints $(y\leq 0$ or $y\geq x)$ is implemented
by the minimum operation. Rewrite it as
%
$$
  y\leq 0\ \lor\ x-y\leq 0.
$$

In other words, at least one of both expressions must be negative, \emph{i.e.},
their minimum must be negative. The corresponding constraint is
%
$$
  \min(y,x-y)\leq 0.
$$
%

The complete program follows.
%
\begin{envcode}
\begin{verbatim}
Constants
  d  = 1.25          ,
  x0 = 1.5, y0 = 0.5 ;

Variables
  int  x in [-5,  +7 ] ,        /* x is an integer variable */
  real y in ]-oo, +oo[ ;        /* y is a real variable (default type) */

Constraints
  d^2 = (x-x0)^2 + (y-y0)^2 ,   /* distance between (x,y) and (x0,y0) */
  min(y, x-y) <= 0          ;   /* y<=0 or y>=x */
\end{verbatim}
\end{envcode}

For this problem, \Softname{} computes three boxes:
%
\begin{envresult}
\begin{verbatim}
OUTER BOX 1
  x = 2
  y in [-0.6456439237389602 , -0.64564392373896]
  precision: 2.22e-16, elapsed time: 0 ms

OUTER BOX 2
  x = 1
  y in [1.64564392373896 , 1.64564392373896]
  precision: 2.22e-16, elapsed time: 0 ms

OUTER BOX 3
  x = 1
  y in [-0.6456439237389602 , -0.64564392373896]
  precision: 2.22e-16, elapsed time: 0 ms

END OF SOLVING
  Property:     reliable process (no solution is lost)
  Elapsed time: 0 ms
\end{verbatim}
\end{envresult}


\index{minimum}
\index{constraint!disjunction}


%------------------------------------------------------------------------------------------
\unnumberedpar{Grammar of Constraints}

\addcontentsline{toc}{subsubsection}{Grammar of Constraints}

\index{constraint!Grammar}
\index{expression}

A constraint is a nonlinear equation or inequation involving numbers,
constants, variables and function symbols. The complete grammar is as
follows:
%
$$
\left[
\begin{array}{rcl}
  \textit{Constraint} & \textsf{::=} & f\,=\, f\mid f\,>=\, f\mid f\,<=\, f\\[2mm]

  f & \textsf{::=} & f+f \mid f-f \mid f*f\mid f\, /\, f\mid\\
    &              & \min(f,f)\mid\max(f,f)\mid\\
    &              &  f\mathop{^\wedge}\textit{natural}\mid\\
    &              & (f)\mid -f\mid +f\mid\\
    &              & \textit{Number}\mid\textit{Constant}\mid\textit{Variable}\mid\\
    &              & \mathop{\textrm{sqrt}(f)}\mid\\
    &              & \log(f)\mid\exp(f)\mid\\
    &              & \cos(f)\mid\sin(f)\mid\tan(f)\mid\\
    &              & \mathrm{acos}(f)\mid\mathrm{asin}(f)\mid\mathrm{atan}(f)\mid\\
    &              & \mathrm{cosh}(f)\mid\mathrm{sinh}(f)\mid\mathrm{tanh}(f)\mid\\
    &              & \mathrm{acosh}(f)\mid\mathrm{asinh}(f)\mid\mathrm{atanh}(f)
\end{array}
\right]
$$
%


%------------------------------------------------------------------------------------------
%------------------------------------------------------------------------------------------
\subsection{Branch-and-Prune Algorithms}
The constraint solving engine of \Softname{} implements a
\emph{branch-and-prune} algorithm. Given a CSP, a set of
boxes that contains all the solutions of the CSP is computed.

\index{branch-and-prune algorithm}
\index{bisection}
\index{splitting}

Each box is reduced and then split. The reduction eliminates
inconsistent values from domains by means of consistency techniques
(see Section~\ref{sec:overview:consistency}). The splitting step
generates sub-boxes in order to separate the solutions.

%------------------------------------------------------------------------------------------
\unnumberedpar{Example: Intersection of Circles}

\addcontentsline{toc}{subsubsection}{Example: Intersection of Circles}

Fig.~\ref{fig:overview:circlesolver} illustrates a branch-and-prune
computation over the aforementioned problem of intersection of
circles. The initial box is $\Ibounds{0}{6}\times\Ibounds{-3}{3}$.

\begin{figure}[htbp]
\hspace*{45mm}\includegraphics[bb=130 610 420 750]{figures/circle-problem-solve.ps}
\caption{Branch-and-Prune Computation.}
\label{fig:overview:circlesolver}
\end{figure}

First, a reduction technique may remove the hashed surface of the
box. The new box is $\Ibounds{0}{3}\times\Ibounds{-3}{3}$.  Suppose
that no more reduction is possible (this depends on reduction
techniques).

Second, the domain of $y$ can be bisected into two parts.  Two sub-boxes
are generated: $\Ibounds{0}{3}\times\Ibounds{-3}{0}$ (box $2$) and
$\Ibounds{0}{3}\times\Ibounds{0}{3}$ (box $3$). Then each sub-box is
reduced, and so on.  The result is the union of both small gray boxes.

Note that a box is bisected only if it is too large with respect to a
desired precision fixed \emph{a priori}. The precision of a box is the
maximum precision componentwise and the precision of an interval
$\Ibounds{a}{b}$ is the quantity $(b-a)$.


%------------------------------------------------------------------------------------------
\unnumberedpar{Paving or Not}

\addcontentsline{toc}{subsubsection}{Paving or Not}

\Softname{} is able to tackle the following problems, given
a natural $n$:
%
\begin{enumerate}
\item Compute at most $n$ boxes that contain all the
   solutions of the CSP.

\item Compute at most $n$ boxes at the desired precision.
\end{enumerate}
%

\index{paving}
\index{continuum of solutions}
\index{discrete solution set}

The \textbf{first problem} deals with the computation of a
\emph{paving} of the solution space.  This mode is useful for
representing continui of solutions by a finite number of boxes. The
maximum number of boxes $n$ is a parameter. The following figure
illustrates a paving of $10$ boxes for the constraint $y\leq x^2-0.8$
on the box $\Ibounds{-1}{1}^2$. The hashed surface represents the set
of solutions.

\begin{figure}[htbp]
\hspace*{55mm}\includegraphics[width=.5\textwidth,bb=130 560 420 750]{figures/bisection-parabola.ps}
\caption{Paving of a continuous set of solutions.}
\label{fig:overview:circlesolve}
\end{figure}

Note that the number of boxes in the paving is also bounded by the
number of boxes of the desired precision that are contained in the
initial box, since the sufficiently tight boxes are not bisected.

The \textbf{second problem} deals with the computation of boxes at the desired
precision. This is typically used for representing discrete sets of
solutions. Note that such a process may not be reliable since some
solutions may be lost due to the maximum number
of computed boxes fixed \emph{a priori}. For instance, for the problem of intersection of circles,
if this number is $1$, \Softname{} returns one box:
%
\begin{envresult}
\begin{verbatim}
OUTER BOX 1
  x in [2.75 , 2.75]
  y in [-2.119516683375301e-16 , +0]

  precision: 4.44e-16, elapsed time: 0 ms

END OF SOLVING
  Property:     non reliable process (some solutions may be lost)
  Elapsed time: 0 ms
\end{verbatim}
\end{envresult}

Is there a difference between both problems when $n=+\infty$?
In both cases, the result is necessarily a paving
composed of boxes at the desired precision. Nevertheless,
the splitting (search) strategy is different:
%
\begin{itemize}
\item A paving must be composed of boxes that have approximatively
   the same precision. To do so, boxes are processed in Breadth-First
   order (BFS), \emph{i.e.}, a First-In First-Out strategy for
   managing the list of boxes.

\item For the second problem, the aim is to compute boxes at the given
  precision as fast as possible. Then boxes are processed in
  Depth-First order (DFS), \emph{i.e.}, a Last-In First-Out strategy.
\end{itemize}
%



%------------------------------------------------------------------------------------------
\unnumberedpar{Programming the Search}

\addcontentsline{toc}{subsubsection}{Programming the Search}

\index{Programming!search}
\index{round-robin}
\index{largest-first}

Optional flags can be added in programs. The first possible action is to
disconnect the splitting process.  In this case, only a reduction
phase is performed.
%
\begin{envcode}
\begin{verbatim}
Split none;
\end{verbatim}
\end{envcode}


Otherwise, if the splitting process is enabled, 
several components of the strategy can be parameterized.
%
\begin{envcode}
\begin{verbatim}
Split  choice    = rr | lf | mn    ,
       parts     = 2 | 3           ,
       precision = p               ,    /* real number */
       mode      = paving | points ,
       number    = n | +oo         ;    /* natural number or +oo */
\end{verbatim}
\end{envcode}
%
There are $3$ choice strategies of bisected domains implemented in the system:
%
\begin{itemize}
\item \texttt{rr} or \texttt{round\_robin}: the variables are chosen according
   to a lexicographic order fixed \emph{a priori}. This is the default
   value.

\item \texttt{lf} or \texttt{largest\_first}: the variable whose domain is the
  largest one (the less precise) is chosen.

\item \texttt{mn} or \texttt{max\_narrow}: the variable associated to the
   maximum absolute column sum norm of the Jacobian matrix is
   chosen. If the Jacobian matrix cannot be computed, then the
   round-robin strategy is used.
\end{itemize}
%
The \texttt{parts} option defines the number of sub-boxes generated at
each splitting step. Only subdivisions in $2$ or $3$ parts are currently
supported.

The \texttt{precision} option corresponds to the precision under which
boxes are not bisected. The default value is set to $10^{-8}$.

The two last options \texttt{mode} and \texttt{number} permit
implementing the two problems handled by \Softname{}: computation of a
paving or not.  In the \texttt{paving} mode, the \texttt{choice}
strategy is necessarily \texttt{largest\_first}.  Note that the default
mode is \texttt{points} (not \texttt{paving}), and the default number
of computed boxes is $1024$.

Finally, it is also possible to stop the search if the computation is
too long. It suffices to add the following flag in programs:
%
\begin{envcode}
\begin{verbatim}
Time = t ;          /* maximum computation time in milliseconds */
\end{verbatim}
\end{envcode}
%

Note that this option may invalidate
Property~\ref{property:overview:complete}: if the resolution is
prematurely stopped, then some solutions may be lost.


%------------------------------------------------------------------------------------------
\unnumberedpar{Example: Brown's problem}

\addcontentsline{toc}{subsubsection}{Example: Brown's problem}

Brown's system is a square quasi-linear system of $n$ equations.
%
$$
\left\lbrace
\begin{array}{rcl}
  n+1 & = & \sum_{i=1}^{n} (x_i) + x_j,\ \ \ j\in\{1,\dots,n-1\}\\[2mm]
  1   & = & \Pi_{i=1}^{n} (x_i)\\[2mm]
  x_i & \in &\Ibounds{-2}{2},\ \ \ i \in\{1,\dots,n\}\\
\end{array}
\right.
$$
%

For $n=5$, \Softname{} with the default mode generates two boxes
(there are $2$ solutions).
%
\begin{envresult}
\begin{verbatim}
OUTER BOX 1
  x[1] in [0.9999999996337671 , 1.000000000366426]
  x[2] in [0.9999999998326983 , 1.000000000153915]
  x[3] in [0.9999999999576185 , 1.000000000035721]
  x[4] in [0.9999999999999982 , 1.000000000000002]
  x[5] in [0.9999999999999919 , 1.000000000000008]

  precision: 7.33e-10, elapsed time: 180 ms
\end{verbatim}
\end{envresult}

\begin{envresult}
\begin{verbatim}
OUTER BOX 2
  x[1] in [0.9163545825338459 , 0.9163545825338514]
  x[2] in [0.9163545825338464 , 0.9163545825338519]
  x[3] in [0.9163545825338466 , 0.9163545825338516]
  x[4] in [0.9163545825338462 , 0.9163545825338518]
  x[5] in [1.418227087330746 , 1.418227087330764]

  precision: 1.82e-14, elapsed time: 460 ms

END OF SOLVING
  Property:     reliable process (no solution is lost)
  Elapsed time: 580 ms
\end{verbatim}
\end{envresult}
%

If the output is required to be the hull of the computed boxes (see
Sec.~\ref{sec:overview:output}), we obtain only one box at the end of
the splitting process.

%
\begin{envresult}
\begin{verbatim}
OUTER BOX: HULL of 2 boxes
  x[1] in [0.9163545825338459 , 1.000000000366426]
  x[2] in [0.9163545825338464 , 1.000000000153915]
  x[3] in [0.9163545825338466 , 1.000000000035721]
  x[4] in [0.9163545825338462 , 1.000000000000002]
  x[5] in [0.9999999999999919 , 1.418227087330764]
  precision: 0.976, elapsed time: 570 ms

END OF SOLVING
  Property:     reliable process (no solution is lost)
  Elapsed time: 570 ms
\end{verbatim}
\end{envresult}
%

If an upper bound for the computation time is set to $300$ms, only one
box is computed, and the property of reliability is no more
guaranteed.
%
\begin{envresult}
\begin{verbatim}
OUTER BOX 1
  x[1] in [0.9999999996337671 , 1.000000000366426]
  x[2] in [0.9999999998326983 , 1.000000000153915]
  x[3] in [0.9999999999576185 , 1.000000000035721]
  x[4] in [0.9999999999999982 , 1.000000000000002]
  x[5] in [0.9999999999999919 , 1.000000000000008]
  precision: 7.33e-10, elapsed time: 180 ms

END OF SOLVING
  Property:     non reliable process (some solutions may be lost)
  Elapsed time: 310 ms
\end{verbatim}
\end{envresult}
%

Using the paving mode, two boxes are computed near the end of the
solving process since a BFS strategy is implemented.
%
\begin{envresult}
\begin{verbatim}
OUTER BOX 1
  x[1] in [0.9999999999974312 , 1.000000000002585]
  x[2] in [0.9999999999985194 , 1.000000000001401]
  x[3] in [0.9999999999994048 , 1.000000000000503]
  x[4] in [0.9999999999999981 , 1.000000000000002]
  x[5] in [0.9999999999999946 , 1.000000000000007]
  precision: 5.15e-12, elapsed time: 550 ms

OUTER BOX 2
  x[1] in [0.9163545825330509 , 0.916354582534548]
  x[2] in [0.9163545825334893 , 0.9163545825341433]
  x[3] in [0.9163545825337194 , 0.9163545825339365]
  x[4] in [0.9163545825338468 , 0.9163545825338524]
  x[5] in [1.41822708733075 , 1.418227087330761]
  precision: 1.5e-12, elapsed time: 550 ms

END OF SOLVING
  Property:     reliable process (no solution is lost)
  Elapsed time: 550 ms
\end{verbatim}
\end{envresult}
%


Combining the paving mode with a maximum computation time of
$300$ms is dangerous for problems having a discrete solution set. In
this case, a paving composed of about $140$ boxes is computed,
though they can be efficiently reduced in about $200$ms.

Is there an ideal strategy? Should we use the paving mode or the
default mode? These aspects are discussed in
Section~\ref{sec:overview:strategy}.


%------------------------------------------------------------------------------------------
%------------------------------------------------------------------------------------------
\subsection{Output of \Softname{}}\label{sec:overview:output}

\index{Programming!output}

The output of \Softname{} can be parameterized by means of optional flags
in programs, as follows.
%
\begin{envcode}
\begin{verbatim}
Output  digits = n                ,      /* natural number */
        mode   = union | hull     ,
        style  = bound | midpoint ;
\end{verbatim}
\end{envcode}
%

The first option corresponds to the number of digits for printing
interval bounds. The default value is set to $16$.

\index{union of boxes}
\index{hull of boxes}

The \texttt{union} mode specifies that the result is presented as the
list of all computed boxes. The \texttt{hull} mode is such that the
result is presented as the hull of all computed boxes, \emph{i.e.},
the smallest box containing all computed boxes.

The \texttt{bound} style is such that every interval domain
$\Ibounds{a}{b}$ is written $\Ibounds{a}{b}$.  In the
\texttt{midpoint} mode, it is written $m + \Ibounds{-e}{e}$ where $m$
is the center of $\Ibounds{a}{b}$ and $e$ is the distance from the
center to each bound.

For instance, let us add the following lines in the program
modeling the intersection of circles.
%
\begin{envcode}
\begin{verbatim}
Output  digits = 32       ,
        mode   = hull     ,
        style  = midpoint ;
\end{verbatim}
\end{envcode}
%

The result follows.
%
\begin{envresult}
\begin{verbatim}
OUTER BOX: HULL of 2 boxes
  x = 2.1999999999999992894572642398998 + [-0.55,+0.55]
  y = 1.1000000000000000888178419700125 + [-1.1,+1.1]

  precision: 2.44e-15, elapsed time: 10 ms

END OF SOLVING
  Property:     reliable process (no solution is lost)
  Elapsed time: 0 ms
\end{verbatim}
\end{envresult}
%



%------------------------------------------------------------------------------------------
%------------------------------------------------------------------------------------------
\subsection{Consistency Techniques}\label{sec:overview:consistency}

\index{consistency technique!local}

A precise descrition of consistency techniques is out of scope of this
manual. However, one may give the user some ideas of the capabilities
of the different algorithms implemented in \Softname{}.

A consistency property determines which values from domains are
consistent or not with respect to the constraints of a CSP.  If the
property is ``reliable'', the inconsistent values do not participate
in any solution. Given a consistency property, a consistency or
reduction technique just eliminates some inconsistent values from domains.

%------------------------------------------------------------------------------------------
\unnumberedpar{Example: Circle}

\addcontentsline{toc}{subsubsection}{Example: Circle}

Consider a circle defined by $x^2 + y^2 = 2$ and the box
$\Ibounds{-2}{4}\times\Ibounds{-1}{1}$. The solution space is the
infinite set of points on the circle that belong to the box (see
Fig.~\ref{fig:overview:circleconsistency}).

\begin{figure}[htbp]
\hspace*{62mm}\includegraphics[bb=130 610 420 700]{figures/circle-consistency.ps}
\caption{Solution Space and Consistency.}
\label{fig:overview:circleconsistency}
\end{figure}

On the one hand, the domain of $y$ cannot be reduced since all the
values in $\Ibounds{-1}{1}$ are consistent, \emph{i.e.}, these values
participate in a solution.

On the other hand, the domain of $x$ can be reduced:
$\IboundsCO{-2}{-\sqrt{2}}\ \cup\ \IboundsOO{-1}{1}\ \cup\ \IboundsOC{\sqrt{2}}{4}$
is the set of inconsistent values. In Fig.~\ref{fig:overview:circleconsistency},
the hashed surfaces represent the corresponding sub-boxes.

\index{arc consistency}

Removing all the inconsistent values from the domain of $x$
can be done by the \emph{arc consistency} technique. However, this is
not reachable over the floating-point numbers since
$\sqrt{2}$ is not exactly represented.
Moreover, it is often more efficient to reduce domains at bounds,
\emph{i.e.}, to keep interval domains. This is done
by \emph{bound consistency} techniques.

\index{bound consistency}

The approximation of bound consistency over interval domains is called
\emph{hull consistency}, and it is implemented in \Softname{} (HC3 and HC4 algorithms).
A reduction step over this problem computes the following box.
%
\begin{envresult}
\begin{verbatim}
OUTER BOX
  x   in [-1.414213562373095 , +1.414213562373095]
  y   in [-1 , +1]
\end{verbatim}
\end{envresult}
%

\index{hull consistency}

For this constraint, hull consistency is perfect: no more reduction at
domain bounds is possible while preserving the solution set. However,
it is very sensitive to the multiple occurrences of variables.
For instance, rewrite the equation as $x\times x + y^2 = 2$.
In this case, \Softname{} with hull consistency does not reduce the initial box.


\begin{envresult}
\begin{verbatim}
OUTER BOX
  x   in [-2 , +4]
  y   in [-1 , +1]
\end{verbatim}
\end{envresult}
%

This problem is handled by \emph{box consistency}, which is also
implemented in \Softname{} (BC3 algorithm). For the circle problem
with $x\times x + y^2 = 2$, box consistency is perfect, as shown below.
%
\begin{envresult}
\begin{verbatim}
OUTER BOX
  x in [-1.414213562373095 , +1.414213562373095]
  y in [-1 , +1]
\end{verbatim}
\end{envresult}
%

\index{box consistency}

The implementation of box consistency is just a search process over
$\Ibold{x}$ that removes the inconsistent values from its bounds. The
depth of the search can be parameterized by the width $\varphi$ of
boxes that are examined: $\varphi=0$ is the best precision. For
instance, the computation for $\varphi=0.05$ is the following. The
resulting box is larger but the computation is faster.
%
\begin{envresult}
\begin{verbatim}
OUTER BOX
  x in [-1.453703703703703 , +1.415020576131686]
  y in [-1 , +1]
\end{verbatim}
\end{envresult}
%

In practice, the combination of hull consistency and box consistency
is efficient. This is automatically done in \Softname{} by Algorithm
BC4.

Given a set of constraints, the domain reductions are iterated until
no domain can be sufficiently reduced. This process, called
\emph{constraint propagation}, can be parameterized by an improvement
factor $q$ interpreted as follows: if the width of a domain is reduced
by a factor less than $q\%$ then it is declared unchanged. If all
domains are declared unchanged then the propagation terminates.

\index{improvement factor}

The factor $q=0$ corresponds to the tightest algorithm. How may $q$ be
tuned? For example, consider the following program that has no
solution.
\begin{envcode}
\begin{verbatim}
Consistency  improve = 0;         /* improvement factor */

Variables    x[1..2] in [-1,1] ;

Constraints  x[2] = x[1]          ,
             x[2] = x[1] + 1.0e-5 ;
\end{verbatim}
\end{envcode}
%
The result follows. \Softname{} detects the unsatisfiability in half a
second using a factor of $0\%$. The solving time is rather long
since this problem is ill-conditionned.
%
\begin{envresult}
\begin{verbatim}
END OF SOLVING
  Property:     no solution in the initial box
  Elapsed time: 500 ms
\end{verbatim}
\end{envresult}

In this case, the reduction must be as tight as possible in order to
limit the combinatorial explosion of search algorithms. For
example, let us try another value, $q=30$.
The computation time is much greater.
%
\begin{envresult}
\begin{verbatim}
END OF SOLVING
  Property:     no solution in the initial box
  Elapsed time: 1,080 ms
\end{verbatim}
\end{envresult}
%
Finally, a factor of $q=50$ leads to the generation of $144$
boxes with a precision of $10^{-8}$ in $1600$ms.
%
\begin{envresult}
\begin{verbatim}
END OF SOLVING
  Property:     reliable process (no solution is lost)
  Elapsed time: 1,600 ms
\end{verbatim}
\end{envresult}
%

However, in practice, it is often desired to slightly weaken the
propagation process. To this end, the default value of the improvement
factor is $10\%$.


%--------------------------------------------------------------------------
\unnumberedpar{Strong Consistency Techniques}

\addcontentsline{toc}{subsubsection}{Strong Consistency Techniques}

\index{consistency technique!strong}

The aforementionned consistency techniques are said to be
\emph{local}. In particular, each reduction is applied
over one domain with respect to one constraint.  Such an approach may be
weak. For example, consider the intersection of two parabolas (see
Fig.~\ref{fig:overview:strongconsistency}).

\begin{figure}[htbp]
\hspace*{54mm}\includegraphics[width=0.5\textwidth,bb=130 650 420 755]{figures/strong-consistency.ps}
\caption{Conjunction of Constraints.}
\label{fig:overview:strongconsistency}
\end{figure}

The aim is to compute a tight box enclosing the solution. However, the
initial box cannot be reduced while removing solutions of $c_1$.
Moreover, it cannot be reduced while removing solutions of $c_2$.  As
a consequence, it cannot be reduced. The problem is that the
conjunction of constraints is not taken as a whole.

The locality problem is handled by strong consistency techniques.
Roughly speaking, local consistency techniques are just combined
with a search algorithm. For the problem of intersection of parabolas,
the domain of $x$ is first bisected (see Fig.~\ref{fig:overview:strongconsistencysolve}).

\begin{figure}[htbp]
\hspace*{54mm}\includegraphics[width=0.6\textwidth,bb=130 630 420 755]{figures/strong-consistency-solve.ps}
\caption{Strong Consistency.}
\label{fig:overview:strongconsistencysolve}
\end{figure}

Then the left-hand box is rejected: it is first reduced with respect
to $c_1$, and the resulting (topmost) gray box contains no solution of
$c_2$.  Moreover, the right-hand box is reduced: the topmost hashed
box contains no solution of $c_1$ and the bottommost hashed box
contains no solution of $c_2$.


\index{3B consistency}

An iteration of such a splitting step permits the computation of a
tight box around the solution. The complete program follows. \Softname{}
implements the so-called 3B consistency technique.
%
\begin{envcode}
\begin{verbatim}
Consistency strong = 3B;

Variables
  x   in [0,2] ,
  y   in [0,2] ;

Constraints
  y = x^2     ,
  y = 2 - x^2 ;
\end{verbatim}
\end{envcode}
%

The result is just the expected box.
%
\begin{envresult}
\begin{verbatim}
OUTER BOX
  x in [0.9999999999999997 , 1]
  y in [0.9999999999999996 , 1]
\end{verbatim}
\end{envresult}
%

The splitting process of 3B consistency iteratively reduces each facet
of a box. In Fig.~\ref{fig:overview:strongconsistencysolve}, the
reduced facet corresponds to the left bound of $x$.  The width $w$ of
the domain of $x$ in the rejected box can be parameterized. The
default value is set to $10^{-3}$. A small value of $w$ leads to a
slow consistency algorithm that computes a tight box.

In \Softname{}, there is another strong consistency technique that we
call weak 3B consistency. It is stronger than local consistencies but
weaker than 3B consistency. Only one reduction of each facet is
performed.

\index{weak 3B consistency}

In practice, should we use a local or a strong consistency technique?
In general, implementing a local consistency is efficient. This is the
default mode of \Softname{}. If the resolution is too long, one may
try another technique (see Section~\ref{sec:overview:strategy}).

There are also some situations where the user aims at computing only
one box enclosing the solution set. A first method is to use the
\texttt{hull} option of \Softname{}. A second method consists in
disabling the splitting process and using a strong consistency.

For example, consider Reimer's system with $3$ variables. This
problem is difficult. Constraint solvers are sensitive to
the width of the initial box.
%
\begin{envcode}
\begin{verbatim}
Variables    x[1..3] in [-100,100] ;

Constraints  x[1]^2 - x[2]^2 + x[3]^2  = 0.5,
             x[1]^3 - x[2]^3 + x[3]^3  = 0.5,
             x[1]^4 - x[2]^4 + x[3]^4  = 0.5 ;
\end{verbatim}
\end{envcode}
%
\Softname{} computes $3$ boxes at $10^{-8}$ in $18,630$ms.
If weak 3B consistency is applied, only $2$ boxes are computed in
$14,670$ms.
%
\begin{envresult}
\begin{verbatim}
OUTER BOX 1
  x[1] in [0.3082486435333776 , 0.308248643533384]
  x[2] in [0.6353521306033765 , 0.6353521306033828]
  x[3] in [0.8992525249461816 , 0.8992525249461838]
  precision: 6.49e-15, elapsed time: 8,710 ms

OUTER BOX 2
  x[1] in [0.8992525249461806 , 0.8992525249461845]
  x[2] in [0.6353521306033739 , 0.6353521306033856]
  x[3] in [0.3082486435333729 , 0.308248643533387]
  precision: 1.42e-14, elapsed time: 8,730 ms

END OF SOLVING
  Property:     reliable process (no solution is lost)
  Elapsed time: 14,670 ms
\end{verbatim}
\end{envresult}
%

%--------------------------------------------------------------------------
\unnumberedpar{Interval Newton}

\addcontentsline{toc}{subsubsection}{Interval Newton}

\index{Interval Newton}
\index{linear relaxation}
\index{preconditionning}
\index{combination}


Given a box, the interval Newton method approximates a square system
of equations with functions from $C^1$ by a first-order Taylor
expansion around the center of the box. A preconditionning of the
computed linear relaxation performs a linear combination of the rows
of the system, which may be very efficient.  Combining the interval
Newton method with consistency techniques leads to powerful
solvers.

The following system is hard for local consistency techniques.
%
$$
\left\lbrace
\begin{array}{l}
x + 2y = 0,\\
x - 2y = 0\\
x, y\in\Ibounds{-1}{1}
\end{array}
\right.
$$
%

We observe the same behaviour for the aforementioned problem of the
intersection of parabolas. The reduction that is computed by a local
consistency technique is weak (if the splitting process is disabled).

\begin{envresult}
\begin{verbatim}
OUTER BOX
  x in [-1 , +1]
  y in [-0.5 , +0.5]

  precision: 2, elapsed time: 0 ms
\end{verbatim}
\end{envresult}
%


The combination of local consistency techniques with the interval
Newton method that is implemented by Algorithm BC5 is efficient. This
is the default mode of \Softname{}, where the interval Newton method is
used only if a square system of equations can be extracted from the
CSP.
%
\begin{envresult}
\begin{verbatim}
OUTER BOX
  x = 0
  y = 0
\end{verbatim}
\end{envresult}
%

In some cases, a box is declared to be safe, \emph{i.e.}, it
certainly contains only one solution. For instance,
let us solve the following problem with BC5.
%
\begin{envcode}
\begin{verbatim}
Variables    x[1..2] in [-1,1] ;

Constraints  x[1]^2 = x[2]       ,
             x[1]^2 + x[2]^2 = 2 ;
\end{verbatim}
\end{envcode}
%
The result follows.
%
\begin{envresult}
\begin{verbatim}
SAFE OUTER BOX 1
  x[1] = 1
  x[2] = 1

SAFE OUTER BOX 2
  x[1] = -1
  x[2] = 1
\end{verbatim}
\end{envresult}
%

\index{safe box}

\newpage

%--------------------------------------------------------------------------
\unnumberedpar{Programming with Consistency Techniques}

\addcontentsline{toc}{subsubsection}{Programming with Consistency Techniques}

\index{Programming!consistency}

In practice, several decisions have to be taken: local or strong consistency?
Which technique? What values for the improvement factor and the splitting
parameter $w$ of strong consistencies?

All these methods can be compared with respect to their
tightness. Roughly speaking, hull consistency is weaker than box
consistency, \emph{etc}.
%
$$
\underbrace{\hbox{hc3}=\hbox{hc4}}_{hull} \leq
\underbrace{\hbox{\ \ bc3\ \ }}_{box} =
\underbrace{\hbox{\ \ bc4\ \ }}_{hull/box} \leq
\underbrace{\hbox{\ \ \ \ \ \ bc5\ \ \ \ \ \ }}_{hull/box/Newton} \leq 
   \hbox{weak3B} \leq \hbox{3B}
$$
%

But the weaker, the faster.

Many local consistency algorithms are implemented in \Softname{}.
There are two reasons: that allows experts in constraint programming
to compare the different techniques, and the best average method in
precision and time for each problem is dynamically chosen by the
hard-coded strategy.


The following piece of code may be added in programs.
%
\begin{envcode}
\begin{verbatim}
Consistency 
     local   = hc3 | hc4 |       /* hull consistency */
               hc4_newton |      /* hull consistency + interval Newton */
               bc3 |             /* box consistency */
               bc3_newton |      /* box consistency + interval Newton */
               bc4 |             /* hull consistency + box consistency */
               bc5         ,     /* hull + box consistency + interval Newton */
     phi     = value       ,     /* precision of box consistency */
     improve = n           ,     /* improvement factor for local consistencies */
     strong  = weak3B | 3B ,     /* weak 3B and 3B consistency */
     width   = w           ;     /* splitting factor for strong consistencies */
\end{verbatim}
\end{envcode}
%

%--------------------------------------------------------------------------
%--------------------------------------------------------------------------
\subsection{Programming the Strategy}\label{sec:overview:strategy}

\index{Strategy}
\index{consistency technique!weakness}
\index{ill-conditionned}

The default solving strategy of \Softname{} applies a local
consistency algorithm to compute at most \texttt{number} boxes at the
given \texttt{precision}: the best average strategy for problems
having discrete sets of solutions. The choice of the local
consistency algorithm is done by the system.

In this mode, two main problems can be encountered:
%
\begin{itemize}
\item Nothing happens! There are two possible reasons:

  \begin{itemize}
  \item The model is ill-conditionned, for example
  $(x^2+y^2=2,\ (x+10^{-8})^2+y^2=2)$. The current version of
  \Softname{} fails. In the future, we plan to implement
  a symbolic module in order to preprocess such systems.

  \item The local consistency technique is too weak (locality
  problem).  A strong consistency like weak 3B consistency or 3B
  consistency may be used. The tuning of the \texttt{width} is very
  important. If the width is too large then no reduction is
  computed. If the width is too small then the solving time is too
  long. We suggest to progressively decrease its value in order to
  detect a threshold for reductions.

  \end{itemize}

\item A huge number of boxes is computed.  There are two possible reasons:

  \begin{itemize}
  \item The model is ill-conditionned\,\dots

  \item A continuum of solutions is bisected. In this case, the
  \texttt{paving} mode should be used.
  \end{itemize}

\end{itemize}

In the \texttt{paving} mode, two parameters have to be tuned: the
precision and the number of computed boxes. However, the tuning may be
hard if the user has no idea of the geometry of the solution set. In
the future, we plan to combine \Softname{} with a graphical interface
in order to represent pavings.

In many situations, the aim is to compute only one box enclosing the
solution set. We propose two different methods.
%
\begin{itemize}
\item The splitting process is disabled, \emph{i.e.},
  the initial box is reduced and the algorithm terminates.  In this
  case, there is often the need for applying a strong consistency
  technique in order to reduce the box the most possible.

\item The paving mode is on and the \texttt{hull} of all the computed
  boxes is returned.
\end{itemize}

For example, consider the following program.

\begin{envcode}
\begin{verbatim}
Variables    x[1..3] in [-10,10];

Constraints  x[1]*x[2]*x[3] = 1,
             x[1]+x[2]+x[3] = 0,
             max(x[1]+x[2],x[2]-x[3]) <= 0;
\end{verbatim}
\end{envcode}

If the splitting process is disabled, the initial box is not reduced.
If weak 3B consistency is applied given a width of $10^{-3}$,
we immediately obtain a small reduction.

\begin{envresult}
\begin{verbatim}
OUTER BOX 1
  x[1] in [-10 , +10]
  x[2] in [-10 , +6]
  x[3] in [-6 , +10]
  Elapsed time: 0 ms
\end{verbatim}
\end{envresult}

If 3B consistency is used  given a width of $10^{-3}$, the reduction
is better but the solving is longer.

\begin{envresult}
\begin{verbatim}
OUTER BOX 1
  x[1] in [-9.989989979949861 , -0.01001002005014042]
  x[2] in [-9.989989979949861 , -0.01001002005014042]
  x[3] in [1.580680000001182 , 10]
  Elapsed time: 2,010 ms
\end{verbatim}
\end{envresult}

If 3B consistency is used  given a width of $10^{-4}$, the solving process
becomes much longer and the reduction is quasi equivalent.

\begin{envresult}
\begin{verbatim}
OUTER BOX 1
  x[1] in [-9.989989979949861 , -0.01001002005014042]
  x[2] in [-9.989989979949861 , -0.01001002005014042]
  x[3] in [1.580936000015381 , 10]
  Elapsed time: 20,520 ms
\end{verbatim}
\end{envresult}

In this case, 3B consistency given a width of $10^{-3}$ seems to be a
good choice. Let us try to compute a paving in the hull mode.
The result follows.

\begin{envresult}
\begin{verbatim}
OUTER BOX: HULL of 1024 boxes
  x[1] in [-9.989989979949861 , -0.01001002005014042]
  x[2] in [-9.989989979949861 , -0.01001002005014042]
  x[3] in [1.372542666573572 , 10]
  Elapsed time: 20 ms
\end{verbatim}
\end{envresult}

The computed box is tight and the solving time is small.  However, the
left bound of $x_3$ is not precise with respect to the results
obtained with a strong consistency technique. We may increase the
number of computed boxes ($1024$ is the default value).

\begin{envresult}
\begin{verbatim}
OUTER BOX: HULL of 99999 boxes
  x[1] in [-9.989989979949861 , -0.01001002005014042]
  x[2] in [-9.989989979949861 , -0.01001002005014042]
  x[3] in [1.569623278118378 , 10]
  Elapsed time: 1,920 ms
\end{verbatim}
\end{envresult}

A number of $10^5$ boxes is not enough. Let us try $10^6$.  In this
case, the result is tight. Note that it is very similar to the one
obtained with 3B consistency.

\begin{envresult}
\begin{verbatim}
OUTER BOX: HULL of 1000000 boxes
  x[1] in [-9.989989979949861 , -0.01001002005014042]
  x[2] in [-9.989989979949861 , -0.01001002005014042]
  x[3] in [1.582747308958421 , 10]
  Elapsed time: 19,100 ms
\end{verbatim}
\end{envresult}


%--------------------------------------------------------------------------
%--------------------------------------------------------------------------
\subsection{Related Work}

Interval analysis has been devised by
Ramon~E.~Moore~\cite{MooreBook1966}. Some interval methods implemented
in \Softname{} have been introduced by
Eldon~R. Hansen~\cite{HansenBook1992}, R.~Baker
Kearfott~\cite{KearfottBook1996} or Arnold
Neumaier~\cite{NeumaierBook1990}.

The framework of continuous CSPs originates from the works of
John~G. Cleary~\cite{ClearyFCS1987} and Ernest
Davis~\cite{DavisAI1987}. These ideas have been developped by many
researchers such as Fr{\'e}d{\'e}ric Benhamou~\cite{BenhamouILPS94},
Alain Colmerauer~\cite{Colmerauer1995}, Boi
Faltings~\cite{FaltingsAI1994}, Timothy
Hickey~\cite{HickeyJFLP2001}, Eero
Hyv{\"{o}}nen~\cite{HyvonenAI1992}, Olivier
Lhomme~\cite{LhommeIJCAI1993}, William Older and Andr{\'e}
Vellino~\cite{OlderVellino1993}, Jean-Fran�ois Puget~\cite{BenhamouICLP99}, Michel
Rueher~\cite{CollavizzaReliable1999}, Maarten Van
Emden~\cite{VanEmdenCO1997} and Pascal Van Hentenryck~\cite{VanHentenryckSIAM1997}.

The main achievement was Numerica~\cite{NumericaBook1997} developed by
Pascal Van Hentenryck
\emph{et al.}, which is a modeling language for global optimization
implementing interval computations and consistency techniques.

Many algorithms implemented in \Softname{} can be found in these early
works on continuous CSPs or in the author's
publications~\cite{BenhamouICLP99,GranvilliersICTAI99,GranvilliersGO2001,GranvilliersRC2001}.
