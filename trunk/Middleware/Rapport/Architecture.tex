\chapter{Architecture}
Dans cette partie nous détaillerons l'architecture de notre application, à travers ses différents points d'extensions.
\begin{figure}[htbp]
  \centering
  \includegraphics[scale=0.70]{img/archi}
  \caption{Architecture de l'outil}
  \label{fig:archi}
\end{figure}

\section{Acquisition}
Le plugin Acquisition consiste à récupérer un fichier vidéo et en extraire une image périodiquement. Le plugin répond aux spécifications fournies par l'interface \verb+IFlux+ (cf. Fig \ref{})
\begin{figure}[htbp]
  \centering
  \includegraphics[scale=0.50]{img/IFlux}
  \caption{Interface IFlux}
  \label{fig:IFlux}
\end{figure}
\section{Reasoning}

\section{NetP}
Le  NetP est dédié à poster des informations sur un  réseau social ou un blog. Un contributeur de ce point d'extension doit être en mesure de proposer une solution pour chaque action que recquière Middleware. Ces actions sont données dans l'interface \verb+IIformation+ suivante :
\begin{figure}[htbp]
  \centering
  \includegraphics[scale=0.50]{img/iinterface}
  \caption{Interface pour NetP}
  \label{fig:IInterface}
\end{figure}

Le contributeur que nous avons implémenté propose d'utiliser le réseau social Facebook et utilise la librairie suivante \cite{restFB}.





