\documentclass[12pt,a4paper,utf8x]{report}
\usepackage [frenchb]{babel}
\usepackage[utf8]{inputenc}  
\usepackage[T1]{fontenc} 
% Pour pouvoir utiliser 
\usepackage{ucs}
\usepackage{textcomp}
\usepackage{graphicx}
\usepackage{keystroke}
\usepackage{amssymb}
\usepackage{amsmath}
%\usepackage{pifont}
\usepackage{url} % Pour avoir de belles url
\usepackage{geometry}
\usepackage{hyperref}


\usepackage {listings}% Pour mettre du code source
\lstset{language=sh}

% Pour pouvoir passer en paysage
\usepackage{lscape}

% Pour pouvoir faire plusieurs colonnes
\usepackage {multicol}

\usepackage{makeidx}% Pour crééer un index
\usepackage{graphicx} % Pour insérer des images (entres autres)
\usepackage{fourier-orns} %logo comme \danger
\usepackage[cc]{titlepic} %rajouter le logo LoCD dans la page de garde
\usepackage{tocbibind}
\usepackage{wasysym} %emoticones

\usepackage{glossaries} % Créer un glossaire

\hypersetup{
  backref=true,
  %permet d'ajouter des liens dans...
  pagebackref=true,%...les bibliographies
  hyperindex=true, %ajoute des liens dans les index.
  colorlinks=true, %colorise les liens
  breaklinks=true, %permet le retour à la ligne dans les liens trop longs
  urlcolor= blue, %couleur des hyperliens
  linkcolor= blue, %couleur des liens internes
  bookmarks=true, %créé des signets pour Acrobat
  bookmarksopen=true,
  %si les signets Acrobat sont créés,
  %les afficher complètement.
  pdftitle={Verif_test_rapport}, %informations apparaissant dans
  pdfauthor={MARGUERITE Alain\\ RINCE Romain},
  %dans les informations du document
  pdfsubject={LoCD}
  %sous Acrobat.
}
%Entête pied de page
%Définition des entêtes : 
\usepackage{fancyhdr}
\pagestyle{fancy}

\lstset{language=XML,    numbers=left
   , tabsize=2
   , frame=single
   , breaklines=true
   , basicstyle=\ttfamily
   , numberstyle=\tiny\ttfamily
   , framexleftmargin=13mm
   , xleftmargin=12mm
   %, frameround={tttt}
   , captionpos=b  }
%\usepackage{pifont}
\usepackage[cc]{titlepic}
\usepackage{url} % Pour avoir de belles url
\usepackage {geometry}

% Pour mettre du code source
\usepackage {listings}
% Pour pouvoir passer en paysage
\usepackage{lscape}

% Pour pouvoir faire plusieurs colonnes
\usepackage {multicol}
% POur crééer un index
\usepackage{makeidx}
\usepackage{graphicx}
\makeindex

% Pour l'interligne de 1.5
\usepackage {setspace}
% Pour les marges de la page
\geometry{a4paper, top=2.5cm, bottom=3.5cm, left=1.5cm, right=1.5cm, marginparwidth=1.2cm}

\parskip=5pt %% distance entre § (paragraphe)
\sloppy %% respecter toujours la marge de droite 

% Pour les pénalités :
\interfootnotelinepenalty=150 %note de bas de page
\widowpenalty=150 %% veuves et orphelines
\clubpenalty=150 

%Pour la longueur de l'indentation des paragraphes
\setlength{\parindent}{15mm}



%%%% debut macro pour enlever le nom chapitre %%%%
\makeatletter
\def\@makechapterhead#1{%
  \vspace*{50\p@}%
  {\parindent \z@ \raggedright \normalfont
    \interlinepenalty\@M
    \ifnum \c@secnumdepth >\m@ne
        \Huge\bfseries \thechapter\quad
    \fi
    \Huge \bfseries #1\par\nobreak
    \vskip 40\p@
  }}

\def\@makeschapterhead#1{%
  \vspace*{50\p@}%
  {\parindent \z@ \raggedright
    \normalfont
    \interlinepenalty\@M
    \Huge \bfseries  #1\par\nobreak
    \vskip 40\p@
  }}
\makeatother
%%%% fin macro %%%%

%Couverture 



\title
{
	\normalsize{ M1 ALMA\\ 
	Université de Nantes\\
	2011-2012}\\
	\vspace{15mm}
	\Huge{Projet de Vérification et Test :\\ TP  évalué}
}



\author{MARGUERITE Alain\\ RINCE Romain
	\vspace{45mm}
}
\titlepic{\includegraphics[scale=1.0]{img/logouniv} }
\date
{	
	\normalsize{Université de Nantes \\ 2 rue de la Houssinière, BP92208, F-44322 Nantes cedex 03, FRANCE
	\\ 
	\vspace{5mm}	
	Encadrant : Mottu Jean-Marie\\
	}
}

\begin{document}
\renewcommand{\labelitemi}{$\bullet$} 	
\maketitle


\clearpage

\tableofcontents
\clearpage

% Pour avoir un interligne de 1,5
\begin{onehalfspace}
\chapter{Introduction}
%Dans le cadre du module Initiation à la recherche, 
L'objectif de ce document est de décrire les notions essentielles à retenir en ce début de projet d'initiation à la recherche. Ces notions font parties d'un même sujet d'étude, au coeurs des travaux de l'équipe OPTI. Elles concernent le sujet des contraintes et des intervalles. Aussi des notions d'optimisation seront abordées.

\section{Méthodes de calculs}
Les méthodes de calcul permettant de résoudre des CSP (Constraint Satisfaction Solveur) ou GCSP (Geometric Constraint Satisfaction Solveur) peuvent être divisées en deux catégories les methodes formelles et les methodes numériques. Après avoir brièvement rappellé les grandes lignes et les outils qui dérivent de ces méthodes..

\subsection{Méthodes formelles}
Elle permettent un une résolution d'un système d'equations. Les variables peuvent être des inconnues complexes réelles ou rationelles. Les approches les plus classiques des méthodes formelles utilisent des théories, telles que les idéaux polynomiaux pour les bases de Gröbner, ou la théorie des déterminant pour la méthode du résultant.  Leurs points forts résident dans leur capacité d'éffectuer des résolutions exactes et complète d'un système d'équation. Les méthodes formelles présentent cependant l'inconveinient de mettre en oeuvre des algorithmes de comlpexité exponentielles.

\subsection{Méthodes numérique}
Capablent de résoudre n'importe quel système d'équation (ou d'égalités) les méthodes numériques présentent cependant un important point faible. Il réside dans les opération d'approximations (d'arrondis)  pouvant entraîner des résultats complétement faux. Cependant ces calculs numériques utilisés  par des méthodes de résolutions par intervalles permettent de contourner ces problèmes. Ainsi nou


blah blahblah blahblah \cite{Jermann} blahblah blahblah blahblah\cite{Goldsztejn} blahblah blah  \cite{Goualard}


dsfsdfsdf\cite{Schichl} dsfsdfsdfsdf \cite{Neumaier}

\chapter{Rapport de Tests}

\section{Dame}
\begin{enumerate}
\item
Erreur peutBouger : absence d'un test dans le if : la dame peut aussi se déplacer sur une ligne. Trouvée grâce au testPeutBougerLigne().
\end{enumerate}

\section{Fou}
\begin{enumerate}
\item
Erreur peutBouger : condition en trop dans le if :le fou peut se déplacer en ligne.  Trouvée grâce au testPeutBougerLigne().
\end{enumerate}

\section{Pion}
\begin{enumerate}
\item
Erreur getType() : retourne chaîne vide au lieu de P.  Trouvée grâce au testGetType().
\end{enumerate}


\section{Echiquier}
\begin{enumerate}
\item
Erreur intervalleLibre : elementAt(1) ou lieu de elementAt(i). Trouvée grâce au testIntervallePasLibre().
\item
Erreur getIntervalle: nom mal choisi. On pourrait croire à une inversion entre le if et le premier elsif. Trouvée grâce au testIntervallePasLibre().
\item
Erreur getLigne : Indice de départ mal positionné. Trouvée grâce au testIntervallePasLibre().
\end{enumerate}

\section{Coup}
\begin{enumerate}
\item
Erreur estValide() : test sur la pièce allant être prise au lieu de celle se déplaçant. Trouvée grâce au test testEstValide().
\item
Erreur annuler() : la pièce prise pouvait être nulle. Trouvée grâce au test testAnnuler().
\end{enumerate}

\section{Partie}
\begin{enumerate}
\item
 Erreur private Joueur joueurs[] : joueurs non initialisés. Trouvée grâce à testPartie().
\item
Erreur echec(Joueur joueur\_p) :  au niveau du nombre de pieces à parcourir. Trouvée grâce à  testJouerEchec().
\item
Erreur changerTrait() : erreur de changement de trait entre les deux joueurs. Touvée grâce à testJouer().

\end{enumerate}




% Pour finir l'interligne de 1,5
\end{onehalfspace}



\printindex

\appendix



\end{document}
