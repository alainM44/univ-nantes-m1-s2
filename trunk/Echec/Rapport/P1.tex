\chapter{Rapport de Tests}

\section{Dame}
\begin{enumerate}
\item
Erreur peutBouger : absence d'un test dans le if : la dame peut aussi se déplacer sur une ligne. Trouvée grâce au testPeutBougerLigne().
\end{enumerate}

\section{Fou}
\begin{enumerate}
\item
Erreur peutBouger : condition en trop dans le if :le fou peut se déplacer en ligne.  Trouvée grâce au testPeutBougerLigne().
\end{enumerate}

\section{Pion}
\begin{enumerate}
\item
Erreur getType() : retourne chaîne vide au lieu de P.  Trouvée grâce au testGetType().
\end{enumerate}


\section{Echiquier}
\begin{enumerate}
\item
Erreur intervalleLibre : elementAt(1) ou lieu de elementAt(i). Trouvée grâce au testIntervallePasLibre().
\item
Erreur getIntervalle: nom mal choisi. On pourrait croire à une inversion entre le if et le premier elsif. Trouvée grâce au testIntervallePasLibre().
\item
Erreur getLigne : Indice de départ mal positionné. Trouvée grâce au testIntervallePasLibre().
\end{enumerate}

\section{Coup}
\begin{enumerate}
\item
Erreur estValide() : test sur la pièce allant être prise au lieu de celle se déplaçant. Trouvée grâce au test testEstValide().
\item
Erreur annuler() : la pièce prise pouvait être nulle. Trouvée grâce au test testAnnuler().
\end{enumerate}

\section{Partie}
\begin{enumerate}
\item
 Erreur private Joueur joueurs[] : joueurs non initialisés. Trouvée grâce à testPartie().
\item
Erreur echec(Joueur joueur\_p) :  au niveau du nombre de pieces à parcourir. Trouvée grâce à  testJouerEchec().
\item
Erreur changerTrait() : erreur de changement de trait entre les deux joueurs. Touvée grâce à testJouer().

\end{enumerate}
