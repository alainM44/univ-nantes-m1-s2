\documentclass[11pt,a4paper,utf8x]{report}
\usepackage [francais]{babel}
\usepackage[utf8]{inputenc}  
\usepackage[T1]{fontenc} 
% Pour pouvoir utiliser 
\usepackage{ucs}
\usepackage{textcomp}
\usepackage{graphicx}
\usepackage{keystroke}
\usepackage{amssymb}
\usepackage{amsmath}
%\usepackage{pifont}
\usepackage{url} % Pour avoir de belles url
\usepackage{geometry}
\usepackage{hyperref}
\usepackage{lscape}%Pour faire des pages en paysage

\usepackage {listings}% Pour mettre du code source
\lstset{language=sh}

% Pour pouvoir passer en paysage
\usepackage{lscape}

% Pour pouvoir faire plusieurs colonnes
\usepackage {multicol}

\usepackage{makeidx}% Pour crééer un index
\usepackage{graphicx} % Pour insérer des images (entres autres)
\usepackage[cc]{titlepic} %rajouter le logo 	 dans la page de garde
\usepackage{tocbibind}
\usepackage{glossaries} % Créer un glossaire
\hypersetup{
  backref=true,
  %permet d'ajouter des liens dans...
  pagebackref=true,%...les bibliographies
  hyperindex=true, %ajoute des liens dans les index.
  colorlinks=true, %colorise les liens
  breaklinks=true, %permet le retour à la ligne dans les liens trop longs
  urlcolor= blue, %couleur des hyperliens
  bookmarks=true, %créé des signets pour Acrobat
  bookmarksopen=true,
  %si les signets Acrobat sont créés,
  %les afficher complètement.
  pdftitle={PG\_Rapport}, %informations apparaissant dans
  pdfauthor={\textsc{Marguerite} Alain, \textsc{Pluchon} Gwenael\\ \textsc{Rincé} Romain, \textsc{Sebari} Nabila},
  %dans les informations du document
  pdfsubject={PG\_Rapport}
  %sous Acrobat.
}
%Entête pied de page
%Définition des entêtes : 
\usepackage{fancyhdr}
\pagestyle{fancy}

\lstset{language=XML,    numbers=left
   , tabsize=2
   , frame=single
   , breaklines=true
   , basicstyle=\ttfamily
   , numberstyle=\tiny\ttfamily
   , framexleftmargin=13mm
   , xleftmargin=12mm
   %, frameround={tttt}
   , captionpos=b  }
%\usepackage{pifont}
\usepackage{url} % Pour avoir de belles url

% Pour mettre du code source
\usepackage {listings}
% Pour pouvoir passer en paysage
\usepackage{lscape}

% Pour pouvoir faire plusieurs colonnes
\usepackage {multicol}
% POur crééer un index
\usepackage{makeidx}
\makeindex

% Pour l'interligne de 1.5
\usepackage {setspace}
% Pour les marges de la page
%\geometry{a4paper, top=2.5cm, bottom=3.5cm, left=1.5cm, right=1.5cm, marginparwidth=1.2cm}

%\parskip=5pt %% distance entre § (paragraphe)
%\sloppy %% respecter toujours la marge de droite 

% Pour les pénalités :
\interfootnotelinepenalty=150 %note de bas de page
\widowpenalty=1500 %% veuves et orphelines
\clubpenalty=1500

%Pour la longueur de l'indentation des paragraphes
%\setlength{\parindent}{15mm}



%%%% debut macro pour enlever le nom chapitre %%%%
\makeatletter
\def\@makechapterhead#1{%
  \vspace*{50\p@}%
  {\parindent \z@ \raggedright \normalfont
    \interlinepenalty\@M
    \ifnum \c@secnumdepth >\m@ne
        \Huge\bfseries \thechapter\quad
    \fi
    \Huge \bfseries #1\par\nobreak
    \vskip 40\p@
  }}

\def\@makeschapterhead#1{%
  \vspace*{50\p@}%
  {\parindent \z@ \raggedright
    \normalfont
    \interlinepenalty\@M
    \Huge \bfseries  #1\par\nobreak
    \vskip 40\p@
  }}
\makeatother

%%%% fin macro %%%%
\hyphenation{a-pé-rio-dique}
\hyphenation{a-pé-rio-diques}
%Couverture 



\title
{
	\normalsize{ M1 ALMA\\ 
	Université de Nantes\\
	2011-2012}\\
	\vspace{15mm}
	\Huge{Rapport de Projet :\\Programmation générative }
}



\author{\textsc{Marguerite} Alain, \textsc{Pluchon} Gwenael\\ \textsc{Rincé} Romain, \textsc{Sebari} Nabila
	\vspace{45mm}
}
%\titlepic{\includegraphics[scale=1.70]{img/logouniv}     \hspace{2cm} \includegraphics[scale=0.12	]{img/logo}}
\date
{	
	\normalsize{Université de Nantes \\ 2 rue de la Houssinière, BP92208, F-44322 Nantes cedex 03, FRANCE
	\\ 
	\vspace{5mm}	
	}
}

\hyphenation{mé-thodes}

\begin{document}
\renewcommand{\labelitemi}{$\bullet$} 	
\maketitle


\clearpage

\tableofcontents
\clearpage

% Pour avoir un interligne de 1,5
\begin{onehalfspace}
\chapter{Introduction}
%Dans le cadre du module Initiation à la recherche, 
L'objectif de ce document est de décrire les notions essentielles à retenir en ce début de projet d'initiation à la recherche. Ces notions font parties d'un même sujet d'étude, au coeurs des travaux de l'équipe OPTI. Elles concernent le sujet des contraintes et des intervalles. Aussi des notions d'optimisation seront abordées.

\section{Méthodes de calculs}
Les méthodes de calcul permettant de résoudre des CSP (Constraint Satisfaction Solveur) ou GCSP (Geometric Constraint Satisfaction Solveur) peuvent être divisées en deux catégories les methodes formelles et les methodes numériques. Après avoir brièvement rappellé les grandes lignes et les outils qui dérivent de ces méthodes..

\subsection{Méthodes formelles}
Elle permettent un une résolution d'un système d'equations. Les variables peuvent être des inconnues complexes réelles ou rationelles. Les approches les plus classiques des méthodes formelles utilisent des théories, telles que les idéaux polynomiaux pour les bases de Gröbner, ou la théorie des déterminant pour la méthode du résultant.  Leurs points forts résident dans leur capacité d'éffectuer des résolutions exactes et complète d'un système d'équation. Les méthodes formelles présentent cependant l'inconveinient de mettre en oeuvre des algorithmes de comlpexité exponentielles.

\subsection{Méthodes numérique}
Capablent de résoudre n'importe quel système d'équation (ou d'égalités) les méthodes numériques présentent cependant un important point faible. Il réside dans les opération d'approximations (d'arrondis)  pouvant entraîner des résultats complétement faux. Cependant ces calculs numériques utilisés  par des méthodes de résolutions par intervalles permettent de contourner ces problèmes. Ainsi nou


blah blahblah blahblah \cite{Jermann} blahblah blahblah blahblah\cite{Goldsztejn} blahblah blah  \cite{Goualard}


dsfsdfsdf\cite{Schichl} dsfsdfsdfsdf \cite{Neumaier}

\chapter{Rapport de Tests}

\section{Dame}
\begin{enumerate}
\item
Erreur peutBouger : absence d'un test dans le if : la dame peut aussi se déplacer sur une ligne. Trouvée grâce au testPeutBougerLigne().
\end{enumerate}

\section{Fou}
\begin{enumerate}
\item
Erreur peutBouger : condition en trop dans le if :le fou peut se déplacer en ligne.  Trouvée grâce au testPeutBougerLigne().
\end{enumerate}

\section{Pion}
\begin{enumerate}
\item
Erreur getType() : retourne chaîne vide au lieu de P.  Trouvée grâce au testGetType().
\end{enumerate}


\section{Echiquier}
\begin{enumerate}
\item
Erreur intervalleLibre : elementAt(1) ou lieu de elementAt(i). Trouvée grâce au testIntervallePasLibre().
\item
Erreur getIntervalle: nom mal choisi. On pourrait croire à une inversion entre le if et le premier elsif. Trouvée grâce au testIntervallePasLibre().
\item
Erreur getLigne : Indice de départ mal positionné. Trouvée grâce au testIntervallePasLibre().
\end{enumerate}

\section{Coup}
\begin{enumerate}
\item
Erreur estValide() : test sur la pièce allant être prise au lieu de celle se déplaçant. Trouvée grâce au test testEstValide().
\item
Erreur annuler() : la pièce prise pouvait être nulle. Trouvée grâce au test testAnnuler().
\end{enumerate}

\section{Partie}
\begin{enumerate}
\item
 Erreur private Joueur joueurs[] : joueurs non initialisés. Trouvée grâce à testPartie().
\item
Erreur echec(Joueur joueur\_p) :  au niveau du nombre de pieces à parcourir. Trouvée grâce à  testJouerEchec().
\item
Erreur changerTrait() : erreur de changement de trait entre les deux joueurs. Touvée grâce à testJouer().

\end{enumerate}

\section{Mise en oeuvre de la méthode de résolution par intervalle}


\subsection{Données en entrée}


\subsection{Algorithmes}
La méthode de résolution par intervalles met en oeuvre deux opérations détaillées dans \cite{Neumaier}: 
\begin{itemize}
\item{Branch}
\begin{quote} Diviser pour mieux régner. \end{quote} Cette méthode consiste à diviser récurcivement un problème en deux sous problèmes. Appliquée à la méthode de résolution par intervalles, on «découpe» en deux l'intervalle concerné (en son milieu ou non). On obtient alors deux problèmes plus «faciles» à étudier.
\item{Bound}
La découpe d'un problème en sous problème peut amener à une situation ou un des sous problème crées ne contient aucune solution. On peut alors étudier trier ces sous problèmes pour supprimer un espaces de recherche superflu. Cette étape est appellée \textbf{pruning}.
\end{itemize}


\subsection{Données en sortie}
 blah blah blah blah blah blah blah blah blah blah blah blah blah blah blah blah blah blah blah blah blah blah blah blah blah blah

\chapter[Librairie d'arbres]{Implémentation d'une librairie sur les arbres%
\chaptermark{Librairie d'arbres}}
Étant donné le retard pris par nos deux binôme dans la réalisation du projet, nous avons préféré effectuer une analyse sur une hiérarchie permettant de représenter les différentes implémentations d'arborescences plutôt que d'essayer de produire du code qui ne serait certainement pas terminé au terme de ce projet.

Dans un premier temps nous proposerons donc un modèle de hiérarchie sur les arborescences qui pourrait permettre, par la suite, une éventuelle implémentation d'une librairie. Puis nous discuterons sur les moyens pouvant être mis en oeuvre pour automatiser le choix des structures.

\section{Une possible hiérarchie sur les arbres%\sectionmark{Hiérarchie}%
}
\sectionmark{Hiérarchie}
\paragraph{Introduction} Dans cette partie, nous allons détailler une hiérarchie d'arborescence qui nous paraît plausible. Pour vous permettre de visualiser cette dernière plus aisément reportez vous à la figure \ref{fig:hierarchie} 
\begin{landscape}
\begin{figure}[h]
	\centering
	\includegraphics[scale=0.5]{Hierarchie_arbres}
	\caption{Une hiérarchie sur les arbres}
	\label{fig:hierarchie}
\end{figure}
\end{landscape}

\paragraph{Justification du modèle} 
\begin{description} 
\item[Arbre k-aire]Dans notre modèle, l'arbre k-aire est la classe/interface père de toutes les autres. Cette appellation n'est peut-être pas la plus pertinente puisqu'il s'agit en fait d'une structure arborescente ou chaque n\oe ud peut avoir $f$ fils et contenir $e$ éléments. Cette structure peut être une interface comme une classe concrète mais elle ne présente que peu d'intêret pour ce dernier cas puisqu'elle n'offre aucun avantage algorithme (complexité etc\dots )
\item[Arbre binaire] Il s'agit d'une spécification de l'arbre k-aire ou le nombre de n\oe uds ne peut contenir qu'un élément et ne posséder que deux fils (en excluant le cas des feuilles). Tout comme l'arbre k-aire, il sera sûrement implémenter comme une interface ou tout du moins comme une classe abstraite.
\item[Tas k-aire] La structure du tas spécifie l'arbre k-aire en forçant le nombre d'éléments dans chaque n\oe ud à un et en ajoutant la relation d'ordre entre la valeur du n\oe ud père et celle de ces fils. Là encore on peut imaginer qu'il s'agira d'une classe abstraite.
\item[B arbre] Le B arbre spécifie l'arbre k-aire en permettant de contenir dans un n\oe ud un nombre d'élément nécessairement compris entre $e$ et $\frac{e}{2}$. De plus le nombre de fils est dépendant du nombre $e'$ d'éléments présents à un moment donné dans le n\oe ud puisqu'il est au maximum de $e'+1$. Il s'agit d'ailleurs d'une classe concrète implémentée dans la première partie du sujet.
\end{description}

Toutes les structures suivantes peuvent être considéré comme étant à implémenter en tant que classes concrètes :
\begin{description} 
\item[Arbre binomial et de Fibonacci] Il s'agit de structures très proches du tas k-aire permettant, respectivement, l'implémentation d'un tas binomial et de Fibonacci.
\item[Tas binaire] Le tas binaire est une implémentation concrète possédant à la fois les propriétés des arbres binaires et celle des tas.
\item[Arbre binaire de recherche] Cette structure de données spécifie l'arbre binaire en imposant une organisation des valeurs dans l'arbre.
\item[AVL] L'AVL est une structure spécifiant l'arbre binaire de recherche en forçant celui-ci à être équilibré.
\item[Arbre rouge-noir] Spécification particulière de l'arbre binaire de recherche.
\item[Arbre 2-3-4] Il s'agit d'une structure de données pouvant être implémentée soit par un B arbre avec le nombre maximal d'élements dans un n\oe ud fixé à 4; soit par un arbre rouge-noir, ce dernier étant isomorphe à l'arbre 2-3-4.
\end{description}

\input{P4}


% Pour finir l'interligne de 1,5
\end{onehalfspace}



\listoffigures

\printindex

\appendix

\bibliographystyle{alpha}
\bibliography{biblio.bib}


\end{document}
