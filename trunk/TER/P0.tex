\section{Introduction }
%Dans le cadre du module Initiation à la recherche, 
L'objectif de ce document est de décrire les notions essentielles à retenir en ce début de projet d'initiation à la recherche. Ces notions font parties d'un même sujet d'étude, au coeurs des travaux de l'équipe OPTI. Elles concernent le sujet des contraintes et des intervalles. Aussi des notions d'optimisation seront abordées.

\section{Méthodes de calculs}
Les méthodes de calcul permettant de résoudre des CSP (Constraint Satisfaction Solveur) ou GCSP (Geometric Constraint Satisfaction Solveur) peuvent être divisées en deux catégories les methodes formelles et les methodes numériques. Après avoir brièvement rappellé les grandes lignes et les outils qui dérivent de ces méthodes..

\subsection{Méthodes formelles}
Elle permettent un une résolution d'un système d'equations. Les variables peuvent être des inconnues complexes réelles ou rationelles. Les approches les plus classiques des méthodes formelles utilisent des théories, telles que les idéaux polynomiaux pour les bases de Gröbner, ou la théorie des déterminant pour la méthode du résultant.  Leurs points forts résident dans leur capacité d'éffectuer des résolutions exactes et complète d'un système d'équation. Les méthodes formelles présentent cependant l'inconveinient de mettre en oeuvre des algorithmes de comlpexité exponentielles.
PARLER DE GERE
\subsection{Méthodes numérique}
blah blah blah blah blah blah blah blah blah blah blah blah blah blah blah blah blah blah 
\subsection{Méthodes de résolution par intervalles} 
blah blahblah blahblah \cite{Jermann} blahblah blahblah blahblah\cite{Goldsztejn} blahblah blah  \cite{Goualard}


dsfsdfsdf\cite{Schichl} dsfsdfsdfsdf \cite{Neumaier}
