\chapter{Conclusion}
% L'étude proposée dans ce document apporte des éléments de réponses à des problèmes de conception majeurs de l'outil de visualisation. La définition d'une boîte et de ses caractéristiques sont notamment précisés. Le programmeur  différentes propositions d'implémentations et une études comparatives des différentes structures de données concrètes. Le document aborde par la suite, les différents problèmes que pose l'implémentations du pavage. Différentes stratégies impliquant notamment différentes combinaisons de structures sont proposées. L'étude pratique qui a été effectuée vient compléter ces différentes propositions\dots    
L'étude proposée dans ce document apporte des éléments de réponses à des problèmes de conception majeurs de l'outil de visualisation. Nous proposons donc une étude des structures permettant l'accès aux éléments d'une boîte, ainsi qu'une étude préliminaire sur les structures stockant le pavage, avec pour objectif de répondre au mieux au cahier des charges. Pour l'étude d'accès aux éléments, différents tests Java ont été effectués, et mettent en avant l'efficacité de l'ArrayList pour stocker ces éléments. Nous avons aussi tenté de proposer une structure efficace pour permettre une visualisation fluide. Le R-tree semble correspondre aux attentes du logiciel, cependant il est toujours nécessaire d'effectuer une seconde étude pour trouver quelle implémentation du R-tree est la plus pertinente.

Bien que nous apportons de nombreux éléments de réponses aux problèmes de conception, il sera encore nécessaire de faire une étude approfondie sur la gestion des filtres, qui a malheureusement due être mis de côté lors de cette étude. Il est en effet nécessaire de définir comment trouver ou organiser les boîtes en fonction des filtres à appliquer. La gestion des filtres a aussi un impact sur la structure de visualisation puisque ceux-ci vont jouer un rôle prépondérant sur l'affichage final (ordre d'affichage, couleur des boîtes\dots). De même, nous n'avons effectué aucune étude sur un moyen de sauvegarder le pavage.
