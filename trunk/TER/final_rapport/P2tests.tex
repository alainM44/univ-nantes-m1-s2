\section{Accès aux éléments d'une boîte}
Dans cette partie nous allons effectuer des tests d'accès aux éléments d'une boîte. Le nombre de coordonnées et de caractéristiques sont les mêmes que dans la partie \ref{chap:Chargement} c'est-à-dire 100 valeurs pour les coordonnées et 20 caractéristiques. Parmi ces coordonnées, sept sont de type \verb+Number+, sept de type \verb+String+ et 6 de type \verb+Interval+. Pour toutes les structures testées, il est nécessaire d'avoir une structure globale contenant les types de chaque caractéristique. Nous effectuerons d'abord des tests d'accès aux caractéristiques puis des tests d'accès aux coordonnées. Nous rappelons aussi que les éléments sont générés aléatoirement durant la création.

Pour plus d'aisance, nous considérerons que les identifiants sont de types \verb+Integer+. Un type \verb+String+ revenant au final au même mais nécessitant au préalable un hachage.

Il est aussi important de préciser que, pour tester, nous avons créé dix-mille boîtes et que pour accéder à une caractéristique, nous accédons d'abord à une boîte aléatoirement. Les boîtes sont stockées dans un tableau de taille statique à accès direct par indice. Même s'il est vrai que cette méthode introduit une constante dans le calcul de la complexité, elle permet de ne pas toujours accéder à la même information et donc de ne pas permettre un accès plus rapide qui ne serait pas réaliste.

Tous les résultats sont répertoriés dans la partie \ref{sec:acces}.

\subsection{Tests d'accès aux caractéristiques}
Pour les tests d'accès aux caractéristiques, nous générons aléatoirement un nombre donnant l'identifiant de la caractéristique.

\paragraph{} Les résultats d'accès pour des boîtes contenant une Map (HashMap ou TreeMap) par type de caractéristique (\verb+Number+, \verb+String+ ou \verb+Interval+) sont indiqués dans le tableau \ref{tab:accesHM}.

On peut voir que les deux structures ne se distinguent pas beaucoup en terme de performance, on observe seulement un écart de moins d'une seconde entre la HashMap et la TreeMap à partir de cent millions d'accès.


\paragraph{} Les résultats d'accès pour des boîtes contenant une Map (HashMap ou TreeMap) pour stocker toutes les caractéristiques sont indiqués dans le tableau \ref{tab:accesHM2}.

Cette fois ci les résultats sont légèrement différents pour les deux structures. La HashMap est en effet plus efficace que le TreeMap de presque 10\%, elle est d'ailleurs plus efficace que lorsque chaque caractéristique était stockée dans sa propre Map.

\paragraph{} Les résultats d'accès pour des boîtes contenant une ArrayList ou une LinkedList pour stocker toutes les caractéristiques sont indiqués dans le tableau \ref{tab:accesAL}.

On peut nettement voir l'efficacité de la structure ArrayList pour les temps d'accès par rapport à la LinkedList, mais aussi par rapport à toutes les autres structures précédemment proposées. Celle-ci étant plus efficace de 40\% à 50\% par rapport aux autres structures. On peut aussi s'étonner des performances de la LinkedList par rapport aux structures Maps, mais on peut supposer que leur proximité est due au faible nombre de caractéristiques à stocker.

\paragraph{} Les résultats d'accès pour des boîtes contenant une ArrayList pour chaque caractéristique, avec une Map globale pour retrouver les indices de chacune de ces caractéristiques dans les boîtes, sont indiqués dans le tableau \ref{tab:accesHMG}.

D'après les résultats, l'utilisation d'une HashMap globale est environ 10\% plus efficace que d'utiliser une TreeMap globale. La structure proposée avec une HashMap globale offre de bonnes performances par rapport aux autres structures proposées précédemment, mais reste tout de même moins efficace qu'une ArrayList stockant toutes les caractéristiques.

\paragraph{Conclusion sur l'accès aux caractéristiques} Les meilleures performances sont obtenues lorsqu'une boîte ne contient qu'une ArrayList pour stocker toutes les caractéristiques. Cependant on peut se demander s'il peut s'avérer utile d'obtenir uniquement les caractéristiques pour un type donné. Si c'est le cas, il sera nécessaire d'effectuer une recherche au préalable pour obtenir les éléments désirés. Tandis que lorsque les boîtes contiennent une ArrayList pour chaque caractéristique, bien que les performances d'accès unique soient légèrement moins bonnes, cette recherche serait alors inutile.

\subsection{Tests d'accès aux coordonnées}
 Les résultats d'accès aux coordonnées pour différentes structures (Map et List) sont détaillés dans le tableau \ref{tab:accesCoord}.

Sans surprise l'ArrayList reste la structure la plus adaptée pour stocker les coordonnées.
