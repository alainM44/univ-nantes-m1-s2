\chapter{Étude des structures de données}
%Dans le cadre du module Initiation à la recherche,  

\section{Introduction}
L'objectif de ce document est de mener une études sur les différentes structures de données nécessaires aux futurs algorithmes de visualisation. Les calculs de compléxité seront réalisés selon le paramètre $n$  représentant le nombre de boîtes, et $d$ est le nombre de dimensions du problème.


\section{Définition de la boîte}
C'est l'entité du pavage. Les accès à ses attributs sont donc cruciaux. On rappelle qu'une boîte est définie de la manière suivante : 
\begin{itemize}
 \item 
Un identifiant : Une String respectant un format précis (c.f 1.1 du document de spécifications).
\item
Une liste de coordonnées. Une liste de d'intervalles de Double.
\end{itemize}

Ces données seront régulièrement requises durant les algorithmes nécessaires à la visualisation. Il est donc important que leurs accès soient rapides, voir direct. Pour le cas de l'identifiant, s'agissant d'une simple string le problème de la structure à utiliser ne se pose pas. Pour la liste des coordonnées en revanche, il s'agit d'une séquence finie de données. Plusieurs possibilités sont alors envisageable : 

\begin{itemize}
 \item
Un tableau : L'accès à une coordonnée serait direct. Les opérations d'ajout et de suppression sont revanches coûteuses. Or il est peu probable que ce type d'opération intervienne.
\item
Une liste : Si l'accès à une coordonnée ne serait pas direct, les opérations d'ajouts de de suppressions sont en temps constant. Ces caractéristiques ne conviennent pas à notre problème.
\item
Une hashmap : Coûteuse si la fonction de hashage n'est pas appropriée, la hashmap propose cependant un accès direct. Cependant même pour un petit nombre d'éléments, son occupation mémoire peut être conséquente. 
\end{itemize}
L'étude ci-dessus des trois structures de données, oriente notre choix vers le tableau. La création d'une boîte a alors une complexité en $O(d)$. De plus l'accès aux différents attributs de la boîte (identifiant, liste des coordonnées) sera en $O(1)$.


\section{Stockage des boîtes}
L'outil de visualisation peut charger un fichier entrée de manière dynamique ou non. Nous nous plaçons ici dans le cadre où cette option de chargement dynamique n'est pas activée. L'outil va lire donc de façon séquentielle chaque ligne du fichier d'entrée. % Le création de structure de  données peut alors se faire de différentes  manières : 
%Chaque boîte est crée de manière séquentielle selon l'ordre du fichier puis stockée dans la structure de donnée. Le trie de la structure aura lieu plus tard ., 
Le nombre potentiellement très grand de boîtes éliminent d'emblée la possibilité de choisir  une table de hashage. En effet même si la fonction de hashage est judicieusement choisie, l'occupation mémoire requise serait bien trop importante. Les listes sont pas appropriées ici. Une complexité en $n²$ pour un accès à une boîte n'est pas raisonnable. ou table de hashage pour le stockage des boîtes. Les arbres ont l'atout de pouvoir stocker et manipuler un grand nombre de d'entités. Les arbres de recherches sont des arborescences ordonnées permettant un accès en $\log(n)$. Dans le cas où la structure serait triée au fur et à mesure de sa construction. Les arbres de recherches proposent de bonnes performances. Par exemple l'utilisation d'un arbre binaire de recherche pour la création de n boîtes aurait une complexité au pire cas   
 






% Les algorithmes  de visualisation vont potentiellement avoir besoin d'accéder à ces différentes valeurs de cette liste. En revanche il est peu probable qu'ils opèrent des modifications (ajouts, suppressions) sur les valeurs de ces listes. Il faut donc privilégier le choix d'une structure concrète performante en terme de lecture comme par exemple un dictionnaire .



\end{itemize}
  



