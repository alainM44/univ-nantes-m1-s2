\chapter{Étude des structures de données}
%Dans le cadre du module Initiation à la recherche,  

\section{Introduction}
 blahblahblahblahblahblahblahblahblah
\section{Définition de la boîte}
C'est l'entité du pavage. Les accès à ses attributs sont donc cruciaux. On rappelle qu'une boîte est définie de la manière suivante : 
\begin{itemize}
 \item 
Un identifiant : Une String respectant un format précis (c.f 1.1 du document de spécifications).
\item
Une liste de coordonnées. Une liste de d'intervalles de Double.
\end{itemize}
Ces données seront régulièrement requises durant les algorithmes nécessaires à la visualisation. Il est donc important que leurs accès soient rapides, voir direct. Pour le cas de l'identifiant, s'agissant d'une simple string le problème de la structure à utiliser ne se pose pas. Pour la liste des coordonnées en revanche, il s'agit d'une séquence finie de données. Les algorithmes  de visualisation vont potentiellement avoir besoin d'accéder à ces différentes valeurs de cette liste. En revanche il est peu probable qu'ils opèrent des modifications (ajouts, suppressions) sur les valeurs de ces listes. Il faut donc privilégier le choix d'une structure concrète performante en terme de lecture comme par exemple un dictionnaire .

\section{Lecture et stockages des boîtes}
L'outil de visualisation peut charger un fichier entrée de manière dynamique ou non. Nous nous plaçons ici dans le cadre où cette option de chargement dynamique n'est pas activée. L'outil va lire donc de façon séquentielle chaque ligne du fichier d'entrée. Le création de structure de  données peut alors se faire de différentes  manières : 
\begin{itemize}
 \item 
Chaque boîte est crée de manière séquentielle selon l'ordre du fichier puis stockée dans la structure de donnée. Le trie de la structure aura lieu plus tard .
\item

\end{itemize}
  



