\chapter{Etude des structures de données}
%Dans le cadre du module Initiation à la recherche,  

\section{Introduction}
 blahblahblahblahblahblahblahblahblah
\section{Définition de la box}
C'est l'entité du pavage. Les accès à ses attributs sont donc cruciaux. On rappelle qu'une boîte est définie de la manière suivante : 
\begin{itemize}
 \item 
Un identifiant : Une String respectant un format précis (c.f 1.1 du document de spécifications).
\item
Une liste de coordonnées. Une liste de d'intervalles de Double.
\end{itemize}
Ces données seront régulièrement requises durant les algorithmes necessaires à la visualisation. Il est donc important que leurs accès soient rapides, voir direct. Pour le cas de l'identifiant, s'agissant d'une simple string le probème de la structure à utiliser ne se pose pas. Pour la liste de coordonnées en revanche, 



