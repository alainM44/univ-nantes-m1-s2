\section{Stockage des boites et visualisation}
Trois problèmes majeurs apparaissent dans la réalisation du logiciel de visualisation. La première est bien entendu la gestion d'une très grande quantité de boites lors de l'affichage. Il est en effet nécessaire d'offrir un accès rapide au informations des boites dans la fenêtre. La seconde est la gestion des filtres sur ces mêmes boites. Et le troisième apparait lors du changement des variables étudiées (changement des dimensions visualisées).

\subsection{Gestion des boites dans l'outil de visualisation}
Une des solutions qui permettrait d'offrir une visualisation fluide du pavage en permettant aisément de répondre au cahier des spécifications serait de représenter le pavages sous une forme de quadtree en 2 dimensions ou octree pour trois dimensions