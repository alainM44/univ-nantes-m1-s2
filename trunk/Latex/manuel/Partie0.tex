\chapter{Introduduction et objectifs}\label{chap:fichDonnees}
\section{Avis au lecteur} 
Ce manuel est destiné à un public désirant utiliser le logiciel LoCD. C'est à dire depuis son installation jusqu'à'à la génération du fichier au format pdf contenant le diagramme désiré. Si une partie est consacrée à la mise en forme de ce fichier de données (\ref{chap:fichDonnees}), nécessaire au fonctionnement de LoCD, ce manuel n'a pas pour objectif d'enseigner les methodes de calculs de ces données statistiques. Les auteurs recommandes l'ouvrage suivant pour un tel apprentissage \cite{stat}. 
Une ligne histoire\glossary{name=histoire,description=aventure}

\section{Présentation du logiciel LoCD}
LoCD permet la création automatique de diagrammes, histogrammes ou nuages de points à partir d'un fichier de données statistiques.  L'outil peut être utilisé de deux manières différentes : en ligne de commande ou par le biais de son interface graphique. Ces deux methodes seront détaillées dans ce manuel. 
%(chapitre \ref{chap:UseGraph})
