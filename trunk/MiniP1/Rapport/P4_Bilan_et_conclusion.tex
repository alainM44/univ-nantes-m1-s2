\chapter{Bilan et conclusion}
\section{Génération des tâches}
Bien que la partie de génération ne fut pas la partie la plus ardue, il a été nécessaire de revenir dessus pour tenir en compte les divers informations et spécifications grapillées sur le sujet auprès des enseignants. Le code à donc souvent été modifié à la volée le rendant petit à petit illisible voire erroné.

On peut aussi regretter que la génération aléatoire de taches entraine l'impossibilité d'étudier leur ordonnancement dans kiwi à cause du problème de l'hyperpériode précédemment évoqué.


\section{Mise en oeuvre des algorithmes}

Les résultats des différents algorithmes sont plutôt satisfaisant. En effet pour chacun d'entre eux les tests de vérifications ont été concluant. Nous regrettons cependant de ne pas avoir eu le temps d'effectuer plus de tests à partir de la génération automatique des tâches.
\'A propos du des algorithmes RMBG et EDFBG. Nous avons rencontré des difficultés à produire un code court et clair. Nous avons perdu beaucoup de temps à débugger un code lourd. Avec du recul, même si les résultats sont positifs, il aurait été peut être été plus judicieux de repartir à zéro et repenser le design de ces deux algorithmes.  L'algorithme EDF-TBS a d'ailleurs été le plus rapide à impléménter vu qu'il a été fait sans se servir de RMBG et EDFBG.

