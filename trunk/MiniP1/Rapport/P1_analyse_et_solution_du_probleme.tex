\chapter{Génération de tâches dans un fichier}
\paragraph{Introduction :}
Dans le cadre du module Systèmes Distribués, l'opportunité de  spécifier et concevoir  un générateur de tâches temps réel sdfsdfsdfsdfsdfsdfsdfsdfsdfsdfsdnous est proposé. Le langage d'implémentation étant libre, notre binome à choisi Java. Ce choix est basé principalement par le fait qu'il s'agissait du langage le plus maitrisé. Cela a permis de concentrer nos efforts sur la mise en place des algorithmes et non sur des probèmes de langage.
 L'objectif est de définir un outil de simulation  d'ordonancement de tâche en temps réel.
La première partie du projet avait pour objectif d'obtenir un nombre n de taches périodiques et apériodiques pour de futurs traitements décrits dans la partie 2 : rajouter lien dym. \'A nouveau le choix du format d'un tel fichier nous était laissé. Nous avons choisis de générer un fichier xml (à nouveau pour des raisons  de simplicité) à la syntaxe suivante : 

\begin{lstlisting}
<genTache.AbstractTache-array>
  <genTache.TachePeriodique>
    <Pi>377</Pi>
    <ri>0</ri>
    <id>1</id>
    <Ci>1</Ci>
    <Di>1</Di>
  </genTache.TachePeriodique>
  <genTache.TachePeriodique>
    <Pi>162</Pi>
    <ri>0</ri>
    <id>2</id>
    <Ci>6</Ci>
    <Di>30</Di>
  </genTache.TachePeriodique>
  <genTache.TacheAperiodique>
    <ri>859</ri>
    <id>3</id>
    <Ci>26</Ci>
    <Di>71</Di>
  </genTache.TacheAperiodique>
</genTache.AbstractTache-array>
\end{lstlisting}
On remarque que les taches périodiques sont identifiées par : 
\begin{itemize}
\item
Pi 
\item
ri 
\item
id 
\item
Ci
\item
Di  
\end{itemize} 
Alors que les tâches apériodiques ont seulement : 
\begin{itemize}
\item
ri 
\item
id 
\item
Ci
\item
Di 
\end{itemize} 

Calcul des tâches ap  
Le calcul des tâches ap est effectué selon la formule suivante : $ U_a =  \frac{\sum_{i=1}^m C_i}{ppcm(P_i)}$   ....  L'utilisateur entre la variable Uap  et le nombre de tache ap qu'il désire (m) les variables restantes 
charge des ap sur une hyperpériode 
