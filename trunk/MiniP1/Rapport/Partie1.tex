\chapter{Génération de tâches dans un fichier}
La première partie du projet avait pour objectif d'obtenir un nombre n de taches périodiques et apériodiques pour de futurs traitements décrits dans la partie 2 : rajouter lien dym. \'A nouveau le choix du format d'un tel fichier nous était laissé. Nous avons choisis de générer un fichier xml (à nouveau pour des raisons  de simplicité) à la syntaxe suivante : 
\begin{lstlisting}
<genTache.AbstractTache-array>
  <genTache.TachePeriodique>
    <Pi>377</Pi>
    <ri>0</ri>
    <id>1</id>
    <Ci>1</Ci>
    <Di>1</Di>
  </genTache.TachePeriodique>
  <genTache.TachePeriodique>
    <Pi>162</Pi>
    <ri>0</ri>
    <id>2</id>
    <Ci>6</Ci>
    <Di>30</Di>
  </genTache.TachePeriodique>
  <genTache.TacheAperiodique>
    <ri>859</ri>
    <id>3</id>
    <Ci>26</Ci>
    <Di>71</Di>
  </genTache.TacheAperiodique>
</genTache.AbstractTache-array>
\end{lstlisting}
On remarque que les taches périodiques sont identifiées par : 
\begin{itemize}
\item
Pi 
\item
ri 
\item
id 
\item
Ci
\item
Di  
\end{itemize} 
Alors que les tâches apériodiques ont seulement : 
\begin{itemize}
\item
ri 
\item
id 
\item
Ci
\item
Di 
\end{itemize} 

