\section{Utilisation par un terminal}
\sectionmark{Terminal}
\label{chap:useterm}

\renewcommand{\labelitemi}{$\bullet$} %changer les puces pour cette page

GT peut être utilisé uniquement en ligne de commande. Cette partie demandes des connaissances pré requises sur les commandes unix. En effet seul les fonctionnalités de l'outil seront explicitées. Le mécanisme des options est similaire à toute autres commandes unix. Pour plus de d'information sur les système \gls{unix}
, nous recommandons l'ouvrage suivant : \cite{linux}.

%%%%%%%%%%%%%%%%%%%%%%%%%%%%%%%%%%%%%%%%%%%%%%%%%%%%%%%%%%%%%%%%%%%%%%%%%

\subsection{Utilisation basique}
\label{sec:usebas}
La simple commande suivante exécutera l'outil GC \verb+java -jar GT+  vous donne la possibilité d'entez un nom de fichier. Vous avez alors deux possibilités : 
\begin{itemize}
\item
Utiliser une démonstration des algorithmes à partir de tâches prédéfinies. Vous devrez par la suite choisir l'ordonnancement que vous souhaitez visualiser. 
\item
Générer vos propres tâches. Suivez les instructions pour la génération de celles ci.
\end{itemize}

%%%%%%%%%%%%%%%%%%%%%%%%%%%%%%%%%%%%%%%%%%%%%%%%%%%%%%%%%%%%%%%%%%%%%%%%%

\subsection{Génération manuelle des tâches}
Il vous sera demandé successivement si vous souhaitez une génération automatique ou des nom des tâches périodiques. Dans le cas d'une génération automatique il est nécessaire de renseigner le pourcentage d'utilisation maximale du processeur. Pour les tâches apériodiques il est aussi nécessaire de fournir leur taux d'occupation du processeur (U\_a).
Dans le cas d'une génération entièrement manuelle, il est demandé de fournir chaque caractéristiques de la tâche. Pour plus de détails sur chaque attribut à fournir nous vous invitons à vous référer au document suivant : \cite{SD}
 
