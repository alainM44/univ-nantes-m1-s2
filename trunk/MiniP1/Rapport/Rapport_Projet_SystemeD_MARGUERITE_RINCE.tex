\documentclass[12pt,a4paper,utf8x]{report}
\usepackage [frenchb]{babel}
\usepackage[utf8]{inputenc}  
\usepackage[T1]{fontenc} 
% Pour pouvoir utiliser 
\usepackage{ucs}
\usepackage{textcomp}
\usepackage{graphicx}
\usepackage{keystroke}
\usepackage{amssymb}
\usepackage{amsmath}
%\usepackage{pifont}
\usepackage{url} % Pour avoir de belles url
\usepackage{geometry}
\usepackage{hyperref}


\usepackage {listings}% Pour mettre du code source
\lstset{language=sh}

% Pour pouvoir passer en paysage
\usepackage{lscape}

% Pour pouvoir faire plusieurs colonnes
\usepackage {multicol}

\usepackage{makeidx}% Pour crééer un index
\usepackage{graphicx} % Pour insérer des images (entres autres)
\usepackage{fourier-orns} %logo comme \danger
\usepackage[cc]{titlepic} %rajouter le logo 	 dans la page de garde
\usepackage{tocbibind}
\usepackage{wasysym} %emoticones
\usepackage{glossaries} % Créer un glossaire
\hypersetup{
  backref=true,
  %permet d'ajouter des liens dans...
  pagebackref=true,%...les bibliographies
  hyperindex=true, %ajoute des liens dans les index.
  colorlinks=true, %colorise les liens
  breaklinks=true, %permet le retour à la ligne dans les liens trop longs
  urlcolor= blue, %couleur des hyperliens
  bookmarks=true, %créé des signets pour Acrobat
  bookmarksopen=true,
  %si les signets Acrobat sont créés,
  %les afficher complètement.
  pdftitle={MiniP1_rapport}, %informations apparaissant dans
  pdfauthor={MARGUERITE Alain\\ RINCE Romain},
  %dans les informations du document
  pdfsubject={LoCD}
  %sous Acrobat.
}
%Entête pied de page
%Définition des entêtes : 
\usepackage{fancyhdr}
\pagestyle{fancy}

\lstset{language=XML,    numbers=left
   , tabsize=2
   , frame=single
   , breaklines=true
   , basicstyle=\ttfamily
   , numberstyle=\tiny\ttfamily
   , framexleftmargin=13mm
   , xleftmargin=12mm
   %, frameround={tttt}
   , captionpos=b  }
%\usepackage{pifont}
\usepackage[cc]{titlepic}
\usepackage{url} % Pour avoir de belles url
\usepackage {geometry}

% Pour mettre du code source
\usepackage {listings}
% Pour pouvoir passer en paysage
\usepackage{lscape}

% Pour pouvoir faire plusieurs colonnes
\usepackage {multicol}
% POur crééer un index
\usepackage{makeidx}
\usepackage{graphicx}
\makeindex

% Pour l'interligne de 1.5
\usepackage {setspace}
% Pour les marges de la page
\geometry{a4paper, top=2.5cm, bottom=3.5cm, left=1.5cm, right=1.5cm, marginparwidth=1.2cm}

\parskip=5pt %% distance entre § (paragraphe)
\sloppy %% respecter toujours la marge de droite 

% Pour les pénalités :
\interfootnotelinepenalty=150 %note de bas de page
\widowpenalty=150 %% veuves et orphelines
\clubpenalty=150 

%Pour la longueur de l'indentation des paragraphes
\setlength{\parindent}{15mm}



%%%% debut macro pour enlever le nom chapitre %%%%
\makeatletter
\def\@makechapterhead#1{%
  \vspace*{50\p@}%
  {\parindent \z@ \raggedright \normalfont
    \interlinepenalty\@M
    \ifnum \c@secnumdepth >\m@ne
        \Huge\bfseries \thechapter\quad
    \fi
    \Huge \bfseries #1\par\nobreak
    \vskip 40\p@
  }}

\def\@makeschapterhead#1{%
  \vspace*{50\p@}%
  {\parindent \z@ \raggedright
    \normalfont
    \interlinepenalty\@M
    \Huge \bfseries  #1\par\nobreak
    \vskip 40\p@
  }}
\makeatother
%%%% fin macro %%%%

%Couverture 



\title
{
	\normalsize{ M1 ALMA\\ 
	Université de Nantes\\
	2010-2011}\\
	\vspace{15mm}
	\Huge{Projet de Travaux pratiques :\\Systèmes Distribués \\ Mini -projet1}
}



\author{MARGUERITE Alain\\ RINCE Romain
	\vspace{45mm}
}
\titlepic{\includegraphics[scale=1.70]{img/logouniv}     \hspace{2cm} \includegraphics[scale=0.12	]{img/logo}}
\date
{	
	\normalsize{Université de Nantes \\ 2 rue de la Houssinière, BP92208, F-44322 Nantes cedex 03, FRANCE
	\\ 
	\vspace{5mm}	
	Encadrant : QUEUDET Audrey \\
	}
}
\newglossaryentry{unix}{name=Unix,description={Le système Unix est un système d'exploitation multi-utilisateurs, multi-tâches, ce qui signifie qu'il permet à un ordinateur mono ou multi-processeurs de faire exécuter simultanément plusieurs programmes par un ou plusieurs utilisateurs. }}

\begin{document}
\renewcommand{\labelitemi}{$\bullet$} 	
\maketitle


\clearpage

\tableofcontents
\clearpage

% Pour avoir un interligne de 1,5
\begin{onehalfspace}
\chapter{Cahier des charges}
\paragraph{Introduction :}
 L'objectif est de définir un outil de simulation  d'ordonnancement de tâches en temps réel. Parmi ses fonctionnalités, l'outil devra pour tester les contraintes temporelles d'un ensemble de tâches générées au préalable. La génération de ces tâches entre dans la conception de l'outil. Cet outil permettra d'exporter le résultat dans un fichier  d'extension $.ktr$ pour être exploité directement par l'outil graphique Kiwi.
 
\section{Données en entrées}
L'outil doit pouvoir permettre à l'utilisateur de rentrer des tâches périodiques et ou apériodiques lui même (en précisant chacun des attributs) ou de demander une génération aléatoire pour les deux catégories.
\section{Fonctionnement}
\begin{itemize}
\item
Une analyse d'ordonnançabilité. L'outil affichera à l'utilisateur les résultats des différents tests avec les conclusions qui en découlent.
\item
Un environnement de simulation. L'outil lors du calcul de l'ordonnancement devra afficher les différents événements. Un bilan de ces action sera résumé dans un fichier au terme de l'exécution (facultatif).

L'outil doit pouvoir proposer plusieurs politiques d'ordonnancement. \`A savoir : 
\begin{itemize}
\item
Pour les tâches périodiques :

\begin{itemize}
\item
Rate Monotonic
\item
EDF
\end{itemize}

\item
Pour les tâches apériodiques : 
\begin{itemize}
\item
BG
\item
TBS
\end{itemize}

\end{itemize} 
\item
Un fichier d'extension $.ktr$ sera généré au terme de l'exécution, et contiendra le déroulement de l'ordonnancement jusqu'à'a son terme.
\item
L'outil doit communiquer, au terme de l'exécution, différents résultats de performance qu'il aura lui même calculés. Les informations à fournir sont les suivantes : 
\begin{itemize}
\item
Le nombre de violations d'échéances.
\item
Le nombre de commutations de contexte et de préemptions.
\end{itemize}
\end{itemize}


\chapter{Analyse et solutions du problème}
\section{Introduction et choix du langage}
Dans le cadre du module Systèmes Distribués, l'opportunité de  spécifier et concevoir  un générateur de tâches temps réel nous est proposé. Le langage d'implémentation étant libre, notre binome a opté pour Java. Ce choix est basé principalement par le fait qu'il s'agissait du langage le plus maitrisé. Cela nous a permis de concentrer nos efforts sur la mise en place des algorithmes et non sur des probèmes de langage.
 L'objectif est de définir un outil de simulation  d'ordonancement de tâche en temps réel.

\section{Modelisation du problème}
Le cahier des charges demandait une gestion de taches périodique et apériodiques. blah blah blah d'ordonancement
\section{Génération de tâches dans un fichier}
La première partie du projet avait pour objectif d'obtenir un nombre n de taches périodiques et apériodiques pour de futurs traitements décrits dans la partie 2 : rajouter lien dym. \'A nouveau le choix du format d'un tel fichier nous était laissé. Nous avons choisis de générer un fichier xml (à nouveau pour des raisons  de simplicité) à la syntaxe suivante : 

\begin{itemize}
\item
Des balises \verb+<genTache.AbstractTache-array>+ encadrent la totalité du fichier.
\item
Une tâche périodique sera définie dans une balise  \verb+<genTache.TachePeriodique>+ 
\item
Une tâche apériodique sera définie dans une balise  \verb+<genTache.TacheAPeriodique>+ 
\item
Dans une tache tous ses attributs seront définis de la manière suivante \verb+<nom_attribut>valeur_attribut</nom_attribut>+
\end{itemize}

Voici un exemple d'un respectant le format décrit ci-dessus : 

\begin{lstlisting}
<genTache.AbstractTache-array>
  <genTache.TachePeriodique>
    <Pi>377</Pi>
    <ri>0</ri>
    <id>1</id>
    <Ci>1</Ci>
    <Di>1</Di>
  </genTache.TachePeriodique>
  <genTache.TachePeriodique>
    <Pi>162</Pi>
    <ri>0</ri>
    <id>2</id>
    <Ci>6</Ci>
    <Di>30</Di>
  </genTache.TachePeriodique>
  <genTache.TacheAperiodique>
    <ri>859</ri>
    <id>3</id>
    <Ci>26</Ci>
    <Di>71</Di>
  </genTache.TacheAperiodique>
</genTache.AbstractTache-array>
\end{lstlisting}
On remarque que les taches périodiques sont identifiées par : 
\begin{itemize}
\item
Pi 
\item
ri 
\item
id 
\item
Ci
\item
Di  
\end{itemize} 
Alors que les tâches apériodiques ont seulement : 
\begin{itemize}
\item
ri 
\item
id 
\item
Ci
\item
Di 
\end{itemize} 
\section{Algorithmes et mises en oeuvre}

Calcul des tâches ap  
Le calcul des tâches ap est effectué selon la formule suivante : $ U_a =  \frac{\sum_{i=1}^m C_i}{ppcm(P_i)}$   ....  L'utilisateur entre la variable Uap  et le nombre de tache ap qu'il désire (m) les variables restantes 
charge des ap sur une hyperpériode 
\section{Interface proposée}


\chapter{Manuel utilisateur}
\chaptermark{Manuel}
%\includegraphics[scale=0.10]{img/logo}
\begin{figure}[htbp]
  \centering
  \includegraphics[scale=0.10]{img/logo}

\end{figure}
\chapter{Introduduction et objectifs}\label{chap:fichDonnees}
\section{Avis au lecteur} 
Ce manuel est destiné à un public désirant utiliser le logiciel LoCD. C'est à dire depuis son installation jusqu'à'à la génération du fichier au format pdf contenant le diagramme désiré. Si une partie est consacrée à la mise en forme de ce fichier de données (\ref{chap:fichDonnees}), nécessaire au fonctionnement de LoCD, ce manuel n'a pas pour objectif d'enseigner les methodes de calculs de ces données statistiques \cite{stat}. %%rajouter biblio 
(chapitre \ref{chap:UseGraph}) 
\section{Présentation du logiciel de création de diagrammes}
LoCD permet la création automatique de diagrammes, histogrammes ou nuages de points à partir d'un fichier de données statistiques.  L'outil peut être utilisé de deux manières différentes : en ligne de commande ou par le biais de son interface graphique. Ces deux methodes seront détaillées dans ce manuel. 

\chapter{Votre premier diagramme}
\section{Introduction}
Ce chapitre va vous permettre de réaliser une vote premier diagramme avec LoCD en moins de 5 minutes  \smiley ! Voici le résultat que vous obtiendrez au terme : 
\begin{figure}[htbp]
  \centering
  \includegraphics[scale=0.60]{img/diagrammenuages}
  \caption{Nuages de points avec toutes les méta données possibles renseignées}
  \label{fig:dnuages}
\end{figure}

\section{1\up{ère} \'Etape : Génération du fichier d'entrée}
Ouvrez un éditeur de texte de votre choix. Saisissez les lignes suivantes (utilisez le copier/coller pour gagner du temps).
\begin{verbatim}
  >TITLE: Les plus grands pays du monde pays (~2010)
  >SUBTITLE: En km²
  >Note: La France n'est que 42ème

  Russie      Canada 	   États-Unis    Chine 	    Brésil 
  17 098 242  9 984 670  9 629 091  	9 596 961   8 514 877 km2 	
\end{verbatim}
Enregistrez le fichier : \verb+mon_premier_diagramme.txt+
\section{2\up{ème} \'Etape : Utilisation de LoCD}
Il ne vous reste plus qu'a lancer la commande suivante :
  \begin{figure}[htbp]
    \centering
    \includegraphics[scale=0.40]{img/ecommandes}
    \caption{Exemple d'utilisation de LoCD en ligne de commandes.}
    \label{fig:ecommandes}
  \end{figure} 
  
Et voilà avez crée votre premier diagramme avec LoCD ! Si vous avez rencontrez des difficultés au cours de ce chapitre, vous pouvez vous référer aux différentes parties de ne manuel qui détaille chaque étapes en détail.
 
\chapter{Installation}
%\label{chap:install}
Lorem ipsum dolor sit amet, consectetur adipiscing elit. Mauris eu dapibus magna. Cras vel elit vel mauris bibendum pulvinar. Lorem ipsum dolor sit amet, consectetur adipiscing elit. Vivamus posuere velit eget mauris volutpat pellentesque. Integer condimentum magna porta enim aliquet fringilla. Lorem ipsum dolor sit amet, consectetur adipiscing elit. Fusce in ante dolor, vel posuere ipsum.

Donec eu augue quam. Pellentesque blandit elementum tellus non feugiat. Donec volutpat lectus elit. Pellentesque imperdiet dui vitae ligula vulputate sit amet congue urna laoreet. Ut nisl ligula, aliquam eu pretium sed, tincidunt et nunc. Pellentesque lacinia venenatis ligula in lobortis. Aliquam lorem lorem, iaculis non lacinia eget, ultricies non dui. Donec ultrices vehicula augue, ut pellentesque massa imperdiet ac. Maecenas feugiat, massa id posuere vestibulum, lectus risus pulvinar metus, ut fermentum neque mauris eu est. Sed posuere venenatis quam sed volutpat. Quisque pellentesque sem ac nulla consequat sagittis. Praesent elementum dolor eget nisi lacinia eget facilisis nisi bibendum. Nulla in urna nisi. Curabitur vitae nisl augue, eget blandit magna. 
%\section{Configuration necessaire)
Lorem ipsum dolor sit amet, consectetur adipiscing elit. Mauris eu dapibus magna. Cras vel elit vel mauris bibendum pulvinar. Lorem ipsum dolor sit amet, consectetur adipiscing elit. Vivamus posuere velit eget mauris volutpat pellentesque. Integer condimentum magna porta enim aliquet fringilla. Lorem ipsum dolor sit amet, consectetur adipiscing elit. Fusce in ante dolor, vel posuere ipsum.

Donec eu augue quam. Pellentesque blandit elementum tellus non feugiat. Donec volutpat lectus elit. Pellentesque imperdiet dui vitae ligula vulputate sit amet congue urna laoreet. Ut nisl ligula, aliquam eu pretium sed, tincidunt et nunc. Pellentesque lacinia venenatis ligula in lobortis. Aliquam lorem lorem, iaculis non lacinia eget, ultricies non dui. Donec ultrices vehicula augue, ut pellentesque massa imperdiet ac. Maecenas feugiat, massa id posuere vestibulum, lectus risus pulvinar metus, ut fermentum neque mauris eu est. Sed posuere venenatis quam sed volutpat. Quisque pellentesque sem ac nulla consequat sagittis. Praesent elementum dolor eget nisi lacinia eget facilisis nisi bibendum. Nulla in urna nisi. Curabitur vitae nisl augue, eget blandit magna. 








\section{Utilisation par un terminal}
\sectionmark{Terminal}
\label{chap:useterm}

\renewcommand{\labelitemi}{$\bullet$} %changer les puces pour cette page

GT peut être utilisé uniquement en ligne de commande. Cette partie demandes des connaissances pré requises sur les commandes unix. En effet seul les fonctionnalités de l'outil seront explicitées. Le mécanisme des options est similaire à toute autres commandes unix. Pour plus de d'information sur les système \gls{unix}
, nous recommandons l'ouvrage suivant : \cite{linux}.

%%%%%%%%%%%%%%%%%%%%%%%%%%%%%%%%%%%%%%%%%%%%%%%%%%%%%%%%%%%%%%%%%%%%%%%%%

\subsection{Utilisation basique}
\label{sec:usebas}
La simple commande suivante exécutera l'outil GC \verb+java -jar GT+  vous donne la possibilité d'entez un nom de fichier. Vous avez alors deux possibilités : 
\begin{itemize}
\item
Utiliser une démonstration des algorithmes à partir de tâches prédéfinies. Vous devrez par la suite choisir l'ordonnancement que vous souhaitez visualiser. 
\item
Générer vos propres tâches. Suivez les instructions pour la génération de celles ci.
\end{itemize}

%%%%%%%%%%%%%%%%%%%%%%%%%%%%%%%%%%%%%%%%%%%%%%%%%%%%%%%%%%%%%%%%%%%%%%%%%

\subsection{Génération manuelle des tâches}
Il vous sera demandé successivement si vous souhaitez une génération automatique ou des nom des tâches périodiques. Dans le cas d'une génération automatique il est nécessaire de renseigner le pourcentage d'utilisation maximale du processeur. Pour les tâches apériodiques il est aussi nécessaire de fournir leur taux d'occupation du processeur (U\_a).
Dans le cas d'une génération entièrement manuelle, il est demandé de fournir chaque caractéristiques de la tâche. Pour plus de détails sur chaque attribut à fournir nous vous invitons à vous référer au document suivant : \cite{SD}
 
 
\chapter{Copyright}
	\paragraph{}
    Ce programme est un logiciel libre : vous pouvez le redistribuer ou
    le modifier selon les termes de la GNU General Public Licence tels
    que publiés par la Free Software Foundation : à votre choix, soit la
    version 3 de la licence, soit une version ultérieure quelle qu'elle
    soit.
	\paragraph{}
    Ce programme est distribué dans l'espoir qu'il sera utile, mais SANS
    AUCUNE GARANTIE ; sans même la garantie implicite de QUALITÉ
    MARCHANDE ou D'ADÉQUATION À UNE UTILISATION PARTICULIÈRE. Pour
    plus de détails, reportez-vous à la GNU General Public License.
	\paragraph	{}
    Vous devez avoir reçu une copie de la GNU General Public License
    avec ce programme. Si ce n'est pas le cas, consultez
 \cite{GNU}   
  \begin{figure}[htbp]
    \centering
    \includegraphics{img/gpl}
  \end{figure}  


\chapter{Jeux d'essais}
Jeux d'essaisJeux d'essaisJeux d'essaisJeux d'essaisJeux d'essaisJeux d'essaisJeux d'essaisJeux d'essaisJeux d'essaisJeux d'essaisJeux d'essaisJeux d'essaisJeux d'essais

\chapter{Bilan et conclusion}
\section{Génération des tâches}
Bien que la partie de génération ne fut pas la partie la plus ardue, il a été nécessaire de revenir dessus pour tenir en compte les divers informations et spécifications grapillées sur le sujet auprès des enseignants. Le code à donc souvent été modifié à la volée le rendant petit à petit illisible voir erroné.

On peut aussi regretter que la génération aléatoire de tâches entraine l'impossibilité d'étudier leur ordonnancement dans kiwi à cause du problème de l'hyperpériode précédemment évoqué \ref{PremPart}.


\section{Mise en oeuvre des algorithmes}
Les résultats des différents algorithmes sont plutôt satisfaisants. En effet pour chacun d'entre eux les tests de vérifications ont été concluants. Nous regrettons cependant de ne pas avoir eu le temps d'effectuer plus de tests à partir de la génération automatique des tâches. Les problèmes rencontrés sur la partie génération automatique (cf section précédente) en est la principale cause.

\`A propos des algorithmes RMBG et EDFBG. Nous avons rencontré des difficultés à produire un code court et clair. Nous avons perdu beaucoup de temps à débugger un code lourd. Avec du recul, même si les résultats sont positifs, il aurait été peut être été plus judicieux de repartir de zéro et repenser le design de ces deux algorithmes.  L'algorithme EDF-TBS a d'ailleurs été le plus rapide à impléménter vu qu'il a été fait sans se servir de RMBG et EDFBG.



% Pour finir l'interligne de 1,5
\end{onehalfspace}

\printglossary

\listoffigures

\printindex

\appendix

\bibliographystyle{alpha}
\bibliography{biblio.bib}


\end{document}
