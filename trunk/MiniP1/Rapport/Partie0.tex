\chapter{Introduction}
\paragraph{Introduction :}
 L'objectif est de définir un outil de simulation  d'ordonancement de tâche en temps réel. Parmies ses fonctionalités, l'outil devra pour tester les contraintes temporelles d'un ensemble de tâches générées au préalable. La génération de ces tâches n'entre par dans la conception de l'outil. Cet outil permettra d'exporter le résultat dans un fichier  d'extension $.ktr$ pour être exploité directement par l'outil graphique Kiwi.
 
\section{Format d'entrée}
Il est définit dans le langage XML. Sa syntase sera la suivante
\subsection{Entête}
\begin{itemize}
\item
Des balises \veb+<genTache.AbstractTache-array>+ encadrent la totalité du fichier.
\item
Une tâche périodique sera définie dans une balise  \veb+<genTache.TachePeriodique>+ 
\item
Une tâche apériodique sera définie dans une balise  \veb+<genTache.TacheAPeriodique>+ 
\item
Dans une tache tous ses attributs seront définis de la manière suivante \veb+<nom_attribut>valeur_attribut</nom_attribut>+
\item
Un commentaire regroupant des informations sur la pavage dans son ensemble (e.g., outil d'obtention, paramètres, temps total, ...)
\end{itemize}
